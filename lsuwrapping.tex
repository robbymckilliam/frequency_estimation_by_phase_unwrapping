%\documentclass[draftcls, onecolumn, 11pt]{../../bib/IEEEtran}
\documentclass[journal]{IEEEtran}

\usepackage{mathbf-abbrevs}

\input{defs}

\begin{document}

\title{Frequency estimation by phase unwrapping}

\author{Robby~G.~McKilliam%
  \thanks{Robby~McKilliam is partly supported by a scholarship from
    the Wireless Technologies Laboratory, CSIRO ICT Centre, Sydney,
    Australia}, Barry~G.~Quinn%
  \thanks{Barry~Quinn is with the Department of Statistics, Macquarie
    University, Sydney, NSW, 2109, Australia}, I.~Vaughan~L.~Clarkson%
  \thanks{Robby~McKilliam and Vaughan~Clarkson are with the School of
    Information Technology \& Electrical Engineering, The University
    of Queensland, Qld., 4072, Australia} and Bill Moran
    \thanks{B. Moran is with the Department of Electrical Engineering and Computer
Science, Melbourne Systems Lab, Dept of Elec \& Electronic Eng, Uni of Melbourne, Vic. 3010, Australia.} }
% The paper headers
\markboth{Robby~G.~McKilliam \emph{et al.}, Frequency estimation by phase unwrapping }%
{DRAFT \today}

% make the title area 
\maketitle

 
\begin{abstract}
  Single frequency estimation is a long-studied problem with application domains
  including radar, sonar, telecommunications, astronomy and medicine.  One
  method of estimation, called \emph{phase unwrapping}, attempts to estimate
  the frequency by performing linear regression on the phase of the received
  signal.  This procedure is complicated by the fact that the received phase
  is `wrapped' modulo $2\pi$ and therefore must be `unwrapped' before the
  regression can be performed.  In this paper we propose an estimator that
  performs phase unwrapping in the least squares sense.  The estimator is
  shown to be strongly consistent and its asymptotic distribution is
  derived. We then show that the problem of computing the least squares phase
  unwrapping is related to a problem in algorithmic number theory known as the
  nearest lattice point problem.  We derive a polynomial time algorithm that
  computes the least squares estimator.  The results of various simulations
  are described for different values of sample size and SNR.
\end{abstract}

\begin{IEEEkeywords}
Frequency estimation, phase unwrapping, central limit theorem, lattices, nearest lattice point problem, number theory
\end{IEEEkeywords}


\section{Introduction}

Estimation of the frequency of a single noisy sinusoid is a long studied
problem with applications including radar, sonar, telecommunications,
astronomy and medicine \cite{Quinn2001, Rife1974}.  In this paper, a single
frequency signal is modelled as a complex sinusoid of the form
\begin{equation} \label{eq_sinusoidal}
A \exp{\big(2\pi j \left(f_0 n + \theta_0 \right)\big)},
\end{equation}
where $f_0$ is the frequency, $\theta_0$ is the phase, $n \in \{1,2\dots,N\}$, $A$
is the signal amplitude and $j = \sqrt{-1}$.  The aim is to estimate the
parameters $f_0$ and $\theta_0$ from the signal
\begin{equation} \label{eq_sigmodel_noise}
v_n = A \exp{\big(2\pi j \left(f_0 n + \theta_0 \right)\big)} + s_n,
\end{equation} 
where the sequence $s_1, s_2, \dots$ is a complex noise process. We shall assume, in
this paper, that the random variables $s_n$ are independent and identically
distributed, and that the distribution of $e^{j \alpha} s_n$ does not depend
on $\alpha$. This will occur exactly when the distribution of $s_n$
depends only on $\vert s_n \vert$.  To ensure identifiability we assume that
$f_0$ and $\theta_0$ are in $[-\nicefrac{1}{2}, \nicefrac{1}{2})$.

The maximum likelihood estimator of frequency under Gaussian white assumptions
is known to be very closely approximated by the frequency that maximises the
\emph{periodogram} (squared magnitude of the Fourier transform) of the $v_n$
\cite{Quinn2001}.
%That is 
%\begin{equation} \label{eq:periodogram}
%\hat{f} = \arg\max_{f\in[0,1)}{ P_v(f) }
%\end{equation}
%where
%\[
%P_v(f) = \frac{1}{N} \left| \sum_{n=1}^{N} v_n e^{2\pi j f n}  \right|^2
%\]
We refer to this estimator as the \emph{periodogram estimator}; its
asymptotic properties have been known for some
time \cite{Quinn2001, Hannan1973, Walker1971}.

Rife and Boorstyn have suggested a practical method for computing the
periodogram estimator, by using the fast Fourier transform to obtain the value
of the periodogram at the Fourier frequencies $f =
-\nicefrac{1}{2},-\nicefrac{1}{2}+\nicefrac{1}{N},-\nicefrac{1}{2}+\nicefrac{2}{N},
\dots, \nicefrac{1}{2}-\nicefrac{1}{N}$ \cite{Rife1974}.  The Fourier
frequency that maximises the periodogram is found and this estimate is then
further refined by a numerical approach such as Newton's method.  A problem
with this approach is that the numerical procedure can fail to locate the
correct maximizer \cite{Rice1988}.  To avoid the problem, Rife and Boorstyn
suggested zero padding the signal to the length $4N$ before performing the
fast Fourier transform. This has recently been shown to work by Quinn \emph{et
  al.}, who also show that applying Newton's method to the derivative of
certain monotonic functions of the periodogram, rather than to the periodogram
itself, ensures that Newton's method will succeed even without any zero
padding \cite{Quinn2008maximizing_the_periodogram}.

Regardless of these implementation difficulties, the periodogram estimator is
widely seen as the best method for single frequency estimation.  It provides
the most accurate results and using the fast Fourier transform can be computed
in only $O(N\log{N})$ arithmetic operations.  Nevertheless, many other methods
for single frequency estimation exist.

One alternative method is \emph{phase unwrapping}~\cite{Fowler2002_freq_est_by_phases}.  Phase unwrapping
estimators appear to have been first suggested by Tretter~\cite{Tretter1985},
who utilized the fact that the phase of a complex sinusoid is a linear
function that is `wrapped' modulo $2\pi$.  %This is
%\[
%\arg\left\{ \exp{\left(2\pi j \left(f_0 n + \theta_0 \right)\right)} \right\} = 2\pi(f_0 n + \theta_0) 
%\]
%modulo $2\pi$.  
If the phase could be `unwrapped' then $f_0$ and $\theta_0$ could be estimated
by linear regression.  Many phase unwrapping estimators have since been
suggested \cite{Kay1989, Clarkson1994AnalysisKaysVariance, Clarkson1999,
  Lovell1991, Lovell1992_stat_perform_inst_freq_est}. A common approach is to compute the first differences of the
arguments of the $v_n$.  The resulting signal then resembles a moving average
process, whose parameters can be estimated by standard linear techniques.
This approach was first suggested by Kay \cite{Kay1989,
  Clarkson1994AnalysisKaysVariance}.  A significant advantage of this approach
is that the moving average process has enough structure for the estimates to
be computed with only $O(N)$ arithmetic operations.  The estimator also
appears to perform well when the signal-to-noise ratio is large.  The major
drawback of Kay's estimator is that it thresholds at a relatively high SNR and
also performs poorly at moderate SNRs \cite{Clarkson1994}. In fact, the
estimator has been shown to be inconsistent, i.e., converges almost surely as
$N \rightarrow \infty$ to the wrong frequency \cite{Quinn2001}.


Another approach is to unwrap the phase in the least squares sense.  We call
the resulting estimator the least squares phase unwrapping estimator (LSPUE).
Clarkson has shown how the LSPUE is related to a problem in algorithmic number
theory known as the nearest lattice point problem \cite{Clarkson1999,
  Agrell2002}.  He has also shown that the LSPUE is closely related to the
periodogram estimator.  Quinn has derived the asymptotic properties of a LSPUE
in the simpler case of phase estimation \emph{i.e.} when $f_0 = 0$ and the aim
is to estimate $\theta_0$ \cite{Quinn2007}.

In this paper we derive the asymptotic properties of the LSPUE for single
frequency estimation (Section \ref{sec:asymptotic_properties}).  We show that
the estimator is strongly consistent and derive its central limit
theorem. Following Clarkson \cite{Clarkson1999} we discuss methods to compute
the LSPUE using a nearest lattice point approach (Section
\ref{sec:LSPUandNP}).  While Clarkson suggested solving the nearest lattice
point problem using the approximate algorithm of Babai \cite{Babai1986}, we
describe a polynomial-time algorithm that computes the estimator exactly.  In
Section \ref{sec:simulations} we provide simulations that display the
statistical performance of the LSPUE alongside the periodogram estimator and
some other single frequency estimators.  The LSPUE proves to be only
marginally less accurate than the periodogram estimator and significantly more
accurate than other estimators based on phase unwrapping. We also use the
simulations to confirm the asymptotic results derived in Section
\ref{sec:asymptotic_properties}.


%\section{Signal model}
%
%A single frequency signal is modelled as a complex sinusoid of the form
%\begin{equation} \label{eq_sinusoidal}
%A \exp{\left(2\pi j \left(f_0 n + \theta_0 \right) \right)}
%\end{equation}
%where $n = 1,2\dots,N$, $f_0$ and $\theta_0$ are in [-\nicefrac{1}{2}, \nicefrac{1}{2}), $A$ is the signal amplitude and $j = \sqrt{-1}$.  The aim is to estimate the parameters $f_0$ and $\theta_0$ from the signal
%\begin{equation} \label{eq_sigmodel_noise}
%v_n = A \exp{\left(2\pi j \left(f_0 n + \theta_0 \right) \right)} + s_n
%\end{equation} 
%where $s_n$ is a complex noise process.


\section{The least squares phase unwrapping estimator}
The argument of the $v_n$, denoted $\angle{v_n}$, is given by
\[
\angle{v_n} = 2\pi\left( f_0 n + \theta_0 + X_n \right) \pmod{2\pi},
\]
where $X_n$ is noise in the range $[-\nicefrac{1}{2}, \nicefrac{1}{2})$ and 
has a distribution depending only on $A$ and the probability density function
(pdf) of the $s_n$.  When $s_n$ is complex Gaussian noise the distribution of
the $X_n$ is known as the projected normal distribution and has been studied
by Mardia and Jupp \cite[p. 46]{Mardia_directional_statistics}.  The
distribution has also been discussed by Quinn \cite{Quinn2007} and Tretter
\cite{Tretter1985}.  Other circular noise distributions may be used, for
example, the wrapped normal, or von Mises distributions \cite{Fisher1993,
  Mardia_directional_statistics}.  In this paper we assume that the $X_n$ are
continuous, independent and identically symmetrically distributed, with
cumulative distribution function (cdf) $F_X\left( x\right) $ and pdf $f_X
\left( x\right)$.  We denote by $\sigma^{2}$ the common variance of the $X_n$.
We also assume that $f_X \left( x\right) $ is unimodal, with mode at
$0$. %Indeed, without such an assumption, the parameter $\theta_0$ is not properly identified.
These assumptions are satisfied by a wide range of circular distributions
including the projected normal, wrapped normal and von Mises distributions
\cite{Fisher1993, Mardia_directional_statistics}

Let $Y_n = \angle{v_n}/ \left(2\pi\right)$.  Then
\begin{equation} \label{eq_polyphasemodel}
Y_n = \fracpart{ f_0 n + \theta_0 + X_n },
\end{equation}
where $\fracpart{x} = x - \round{x}$ is the fractional part of $x$ and $\round{x}$
denotes the nearest integer to $x$\footnote{The direction of rounding for
  half-integers is not important, as long as it is consistent. The authors
  have chosen to round up half-integers here.}.

We may write \eqref{eq_polyphasemodel} as
\begin{equation} \label{eq:eq_polyphasemodel}
Y_n = f_0 n + \theta_0 + X_n + U_n,
\end{equation}
where $U_n = -\round{f_0 n + \theta_0 + X_n}$.  By considering the $U_n$ as
nuisance parameters, where $U_n \in \ints$, we may derive the sum of squares
function
\begin{equation} \label{eq:sumofsquaresfunctionwithU}
\sum_{n=1}^{N}\left(  Y_{n} - \theta - fn - U_{n}\right)^{2}.
\end{equation}
For fixed $f$ and $\theta$, \eqref{eq:sumofsquaresfunctionwithU} is minimized
when $U_{n} = -\round{Y_{n} - \theta - fn}$.  Substituting this into
\eqref{eq:sumofsquaresfunctionwithU} we obtain
\begin{equation} \label{eq:sumofsquaresfunction}
SS(f, \theta) = \sum_{n=1}^{N}\fracpart{  Y_{n} - \theta - fn }^{2}.
\end{equation}
The LSPUE returns the $f$ and $\theta$ that minimise $SS(f, \theta)$.


%%% Local Variables: 
%%% mode: latex
%%% TeX-master: "lsuwrapping.tex"
%%% End: 



\section{Asymptotic properties} \label{sec:asymptotic_properties}

In this section the LSPUE is shown to be strongly consistent and its central
limit theorem is derived.  The main result is Theorem \ref{thm:asymp_proof},
the proof of which is given at the end of this section.  In what follows, order
notation will always refer to behaviour as $N \rightarrow \infty$, and the
notation $O_P \left( \cdot \right)$ will mean order in probability as $N
\rightarrow \infty$.

\begin{theorem} \label{thm:asymp_proof} Let $\left(
    \widehat{f}_{N},\widehat{\theta}_{N}\right) $ be the minimiser of
  \eqref{eq:sumofsquaresfunction} over $\left[
    \nicefrac{-1}{2},\nicefrac{1}{2}\right)^2$. Then, $N\left(\widehat{f}_{N}-f_{0}\right)$ and $\widehat{\theta}_{N}-\theta_{0}$ converge almost surely to $0$ as $N\rightarrow\infty,$ and the distribution of
\[
\left[
\begin{array}
[c]{cc}%
N^{3/2}\left(  \widehat{f}_{N}-f_{0}\right)  & N^{1/2}\left( \widehat{\theta}_{N}-\theta_{0} \right)%
\end{array}
\right]  ^{\prime}%
\]
converges to the normal with mean $0$ and covariance matrix%
\[
\frac{12\sigma^{2}}{\left(  1-h\right)  ^{2}}\left[
\begin{array}
[c]{cc}%
1 & -\nicefrac{1}{2}\\
-\nicefrac{1}{2} & \nicefrac{1}{3}
\end{array}
\right]  ,
\]
where $\sigma^{2}=\operatorname{var}X_{n}$ and $h=f_X \left(-1/2\right)$.
\end{theorem}

Substituting $\eqref{eq_polyphasemodel}$ into $SS\left( f,\theta\right) $ we
obtain
\begin{align*}
&  \sum_{n=1}^{N}\fracpart{ \fracpart{ \theta_{0} + f_{0}n + X_{n} } - \theta - fn }
^{2}\\
&  =\sum_{n=1}^{N}\fracpart{  X_{n}+(f_0 - f)n + (\theta_0-\theta) }  ^{2}\\
&  =\sum_{n=1}^{N}\fracpart{  X_{n}+\lambda n+\phi }  ^{2}\\
&  =NS_{N}\left(  \lambda,\phi\right)  ,
\end{align*}
say, where $\lambda=f_{0}-f$ and $\phi=\theta_{0}-\theta$.  Note that $S_N$ is
periodic with period $1$ in both $\lambda$ and $\phi$.  We may thus assume
that $(\lambda, \phi) \in B$ where $B = [-\nicefrac{1}{2},
\nicefrac{1}{2})^2$. Let $\widehat{\lambda}_{N}$ and $\widehat{\phi}_{N}$ be
the minimisers of $S_{N}\left( \lambda,\phi\right)$.  We shall show that
$N\widehat{\lambda}_{N}\rightarrow0$ and $\widehat{\phi}_{N}\rightarrow 0$
almost surely as $N\rightarrow\infty$.  Put
\begin{align*}
V_{N}\left(  \lambda,\phi\right)   &  =S_{N}\left(  \lambda,\phi\right)
-ES_{N}\left(  \lambda,\phi\right) \\
&  =\frac{1}{N}\sum_{n=1}^{N}\left(  \fracpart{  X_{n}+\lambda n+\phi }
^{2}-E\fracpart{  X_{n}+\lambda n+\phi }  ^{2}\right)  .
\end{align*}
Let $(\lambda_j, \phi_k)$ denote the point $\left(  \frac{j}{N^{b+1}} - \nicefrac{1}{2},\frac{k}{N^{b}} - \nicefrac{1}{2}\right)$, for some $b>0$ and let
\begin{align*}
B_{jk}= \left\{ \left(  x,y\right)  ; \frac{j}{N^{b+1}}\leq x + \nicefrac{1}{2}<\frac
{j+1}{N^{b+1}}, \right. \\ \left. \frac{k}{N^{b}}\leq y + \nicefrac{1}{2} <\frac{k+1}{N^{b}} \right\}  .
\end{align*}
Then%
\begin{align*}
&\sup_{\left(  \lambda,\phi\right)  \in B}\left\vert V_{N}\left(  \lambda
,\phi\right)  \right\vert \\  &=\sup_{j,k}\sup_{\left(  \lambda,\phi\right)
\in B_{jk}}\left\vert V_{N}\left(  \lambda_{j},\phi_{k}\right) + V_{N}\left(
\lambda,\phi\right)  -V_{N}\left(  \lambda_{j},\phi_{k}\right)  \right\vert \\
&  \leq\sup_{j,k}\left\vert V_{N}\left(  \lambda_{j},\phi_{k}\right)
\right\vert \\
&+\sup_{j,k}\sup_{\left(  \lambda,\phi\right)  \in B_{jk}%
}\left\vert V_{N}\left(  \lambda,\phi\right)  -V_{N}\left(  \lambda_{j}%
,\phi_{k}\right)  \right\vert ,
\end{align*}
where $j = 0,1,\ldots,N^{b+1}-1$ and $k=0,1,\ldots,N^{b}-1$.

\begin{lemma} \label{lem:supVjk}
$\sup_{j,k}\left\vert V_{N}\left(  \lambda_{j},\phi_{k}\right)  \right\vert
\rightarrow0,$ almost surely as $N\rightarrow\infty.$
\end{lemma}

\begin{IEEEproof}
For any $\varepsilon>0$,
\begin{align*}
P\left(  \sup_{j,k}\left\vert V_{N}\left(  \lambda_{j},\phi_{k}\right)
\right\vert >\varepsilon\right)   &  \leq\sum_{j,k}P\left(  \left\vert
V_{N}\left(  \lambda_{j},\phi_{k}\right)  \right\vert >\varepsilon\right) \\
&  \leq\sum_{j,k}\frac{E\left(  V_{N}^{2\beta}\left(  \lambda_{j},\phi
_{k}\right)  \right)  }{\varepsilon^{2\beta}},
\end{align*}
by Markov's inequality, for any $\beta>0.$ Let%
\[
E\left(  V_{N}^{2\beta}\left(  \lambda,\phi\right)  \right)  =\frac
{1}{N^{2\beta}}E\left(  \sum_{n=1}^{N}Z_{n}\right)  ^{2\beta},
\]
where each of%
\[
Z_{n}=\fracpart{  X_{n}+\lambda n+\phi }  ^{2}-E\fracpart{  X_{n}+\lambda
n+\phi }  ^{2}%
\]
has mean $0$. In Lemma \ref{lem:zero_mean_indepent_sum_bound} in the appendix we show, for integers $\beta$, that
\begin{equation}
E\left(  \sum_{n=1}^{N}Z_{n}\right)  ^{2\beta}= O\left(N^\beta\right) . 
\label{eq:markapp}%
\end{equation}

%\begin{equation}
%E\left(  \sum_{n=1}^{N}Z_{n}\right)  ^{2\beta}=\sum_{n_{1},\ldots,n_{2\beta
%}=1}^{N}E\left(  \prod\limits_{j=1}^{2\beta}Z_{n_{j}}\right)  .
%\label{eq:markapp}%
%\end{equation}
%Since the terms%
%\[
%E\left(  \prod\limits_{j=1}^{2\beta}Z_{n_{j}}\right)
%\]
%in this containing any $n_{j}$ only once will have $0$ mean, and since the
%$Z_{j}$ are bounded, it follows that%
%\[
%\frac{1}{N^{\beta}}E\left(  \sum_{n=1}^{N}Z_{n}\right)  ^{2\beta}=O\left(
%1\right)  ,
%\]
%as $N\rightarrow\infty,$ the dominant terms coming from the $Z_{j}$ in
%$\left(  \ref{eq:markapp}\right)  $ occurring in pairs. 
Hence%
\[
P\left(  \sup_{j,k}\left\vert V_{N}\left(  \lambda_{j},\phi_{k}\right)
\right\vert >\varepsilon\right)  =O\left(  N^{2b+1-\beta}\right)  ,
\]
and, since for any $b>0,$ we may choose $\beta$ so that $2b+1-\beta<-1,$ it
follows that
\[
\sum_{N=1}^{\infty}P\left(  \sup_{j,k}\left\vert V_{N}\left(  \lambda_{j}%
,\phi_{k}\right)  \right\vert >\varepsilon\right)  <\infty,
\]
and consequently from the Borel-Cantelli lemma that%
\[
\sup_{j,k}\left\vert V_{N}\left(  \lambda_{j},\phi_{k}\right)  \right\vert
\rightarrow 0,
\]
almost surely as $N\rightarrow\infty.$
\end{IEEEproof}

\begin{lemma}
$\sup_{j,k}\sup_{\left(  \lambda,\phi\right)  \in B_{jk}}\left\vert
V_{N}\left(  \lambda,\phi\right)  -V_{N}\left(  \lambda_{j},\phi_{k}\right)
\right\vert \rightarrow0,$ almost surely as $N\rightarrow\infty.$
\end{lemma}

\begin{IEEEproof}
Let $a_n = X_{n}+\lambda n+\phi$ and $b_n = X_{n}+\lambda_{j}n+\phi
_{k}$. For $\left(  \lambda,\phi\right)  \in B_{jk},$%
\[
\fracpart{  a_n }  =\fracpart{  b_n +\nu_{n} }  ,
\]
where 
\[
0 \leq\nu_{n} = a_n - b_n = \left(  \lambda-\lambda_{j}\right)  n+\left(  \phi-\phi_{k}\right) \leq2N^{-b}.
\] 
Thus
\[
\fracpart{  a_n }  =\fracpart{  b_n }  +\nu_{n}-\delta_{n},
\]
where%
\[
\delta_{n}=\left\{
\begin{array}
[c]{ccc}%
0 & ; & \fracpart{  b_n }  +\nu_{n}<1/2\\
1 & ; & \text{otherwise.}%
\end{array}
\right.
\]
Hence%
\begin{align*}
\fracpart{  a_n }  ^{2}-\fracpart{  b_n }^{2}
& =\left(  \nu_{n}-\delta_{n}\right)  ^{2}+2\left(  \nu_{n}-\delta
_{n}\right)  \fracpart{ b_n } \\
& =\nu_{n}^{2}+2\nu_{n}\fracpart{  b_n }
-\delta_{n}\left[  2\left(  \fracpart{  b_n }
+\nu_{n}\right)  -1\right]  ,
\end{align*}
since $\delta_{n}^{2}=\delta_{n}.$ Now, if $\delta_{n}=1,$%
\[
0  \leq 2\left(  \fracpart{  b_n }  +\nu_{n}\right)  -1  \leq2\nu_{n} \leq 4N^{-b}.
\]
Thus%
\[
0 \leq\frac{1}{N}\sum_{n=1}^{N}\delta_{n}\left[  2\left(  \fracpart{
b_n }  +\nu_{n}\right)  -1\right] \leq4N^{-b},
\]
and so%
\[
0  \leq\frac{1}{N}E\sum_{n=1}^{N}\delta_{n}\left[  2\left(  \fracpart{
b_n }  +\nu_{n}\right)  -1\right] 
 \leq4N^{-b}.
\]
Also%
\begin{align*}
& \left\vert \nu_{n}^{2}+2\nu_{n}\fracpart{  b_n }  -E\left[  \nu_{n}^{2}+2\nu_{n}\fracpart{  b_n }  \right]  \right\vert \\
&  \leq2\nu_{n}\left\vert \fracpart{  b_n }
\right\vert +2E\nu_{n}\left\vert \fracpart{  b_n }
\right\vert \\
&  \leq4N^{-b}.
\end{align*}
Hence%
\[
\sup_{\left(  \lambda,\phi\right)  \in B_{jk}}\left\vert V_{N}\left(
\lambda,\phi\right)  -V_{N}\left(  \lambda_{j},\phi_{k}\right)  \right\vert
\leq12N^{-b},
\]
and the result follows as the bound does not depend on $j$ or $k$.
\end{IEEEproof}

\begin{theorem}
\label{th:vn}$\sup_{\lambda,\phi}\left\vert V_{N}\left(  \lambda,\phi\right)
\right\vert \rightarrow0,$ almost surely as $N\rightarrow\infty.$
\end{theorem}

\begin{IEEEproof}
The proof follows immediately from the two previous lemmas.
\end{IEEEproof}

\begin{lemma}
\label{lem:one} Let $g:[-\nicefrac{1}{2}, \nicefrac{1}{2})\rightarrow \reals$ be given by
\[
g\left(  x\right)  =E\fracpart{  X_{n}+x }  ^{2}-\sigma^{2}.
\]
Then $g\left(  x\right)  \geq0,$ with equality if and only if $
x  = 0.$
\end{lemma}

\begin{IEEEproof}
If $x\geq0,$%
\begin{align*}
g\left(  x\right)   & =\int_{-1/2}^{1/2-x}\left(  x+y\right)  ^{2}f_X\left(
y\right)  dy \\
&+\int_{1/2-x}^{1/2}\left(  x+y-1\right)  ^{2}f_X \left(  y\right)
dy-\int_{-1/2}^{1/2}y^{2}f_X \left(  y\right)  dy\\
&  =x^{2}+\int_{1/2-x}^{1/2}\left(  1-2x-2y\right)  f_X \left(  y\right)  dy,
\end{align*}
since $E\left(  X\right)  =0.$ Similarly, when $x<0,$%
\[
g\left(  x\right)  =x^{2}+\int_{-1/2}^{-1/2-x}\left(  1+2x+2y\right)
f_X \left(  y\right)  dy  = g\left(  -x\right)
\]
and therefore $g$ is even. Now, since $f_X \left(  y\right)  $ is even, when $x\geq0,$%
\begin{align*}
g^{\prime}\left(  x\right)   &  =2x-2\left[  1-F_X \left(  \frac{1}{2}-x\right)
\right] \\
&  =2x-2F_X \left(  x-\frac{1}{2}\right).
\end{align*}
Since $f_X$ is symmetric and unimodal with mode at $0$, $F_X
(-\nicefrac{1}{2})=0$, $F_X(0)=\nicefrac{1}{2}$ and $F_X(x - \nicefrac{1}{2})$
is strictly convex on $[0, \nicefrac{1}{2})$.  It follows that $g$ is
monotonically increasing on $[0, \nicefrac{1}{2})$ and, being even, is
monotonically decreasing on $[-\nicefrac{1}{2}, 0)$.  Also $g(0) = 0$.  Thus
$g\left( x\right) \geq0,$ with equality if and only if $x=0.$
\end{IEEEproof}

Let
\begin{equation}
\left(  \widehat{\lambda}_{N},\widehat{\phi}_{N}\right)  =\arg\min
S_{N}\left(  \lambda,\phi\right)  \label{eq:lsest}%
\end{equation}
and put $\tau_{N}\left(  \lambda,\phi\right)  =ES_{N}\left(  \lambda
,\phi\right)  .$ Now,%
\[
\tau_{N}\left(  \lambda,\phi\right)  -\sigma^{2}  =\frac{1}{N}\sum
_{n=1}^{N}g\left(  \fracpart{  \lambda n+\phi }  \right)  \geq 0.
\]
Thus, since%
\[
S_{N}\left(  \widehat{\lambda}_{N},\widehat{\phi}_{N}\right)  \leq
S_{N}\left(  0,0\right)  ,
\]
we have%
\[
V_{N}\left(  \widehat{\lambda}_{N},\widehat{\phi}_{N}\right)  +\tau_{N}\left(
\widehat{\lambda}_{N},\widehat{\phi}_{N}\right)  \leq V_{N}\left(  0,0\right)
+\sigma^{2},
\]
and so%
\[
0  \leq\tau_{N}\left(  \widehat{\lambda}_{N},\widehat{\phi}_{N}\right)
-\sigma^{2} \leq-V_{N}\left(  \widehat{\lambda}_{N},\widehat{\phi}_{N}\right)
+V_{N}\left(  0,0\right)  .
\]
However, from Theorem \ref{th:vn}, the right hand side converges almost surely
to $0$ as $N\rightarrow\infty.$ Hence%
\begin{equation}
\tau_{N}\left(  \widehat{\lambda}_{N},\widehat{\phi}_{N}\right)  -\sigma
^{2}\rightarrow0, \label{eq:taucon}%
\end{equation}
almost surely as $N\rightarrow\infty.$

\begin{lemma}
\label{lem:moran}Let $K$ be a subset of the integers $W_{N}=\left\{
1,2,\ldots,N\right\}  $ for which $\#K>3N/4$ where $\#K$ is the cardinality of $K$. Then~\footnote{The notation $K-K$ should not be confused with set subtraction.}
\[
W_{N/2}\subset K-K=\left\{  k-k^{\prime};k,k^{\prime}\in K\right\}.
\]

\end{lemma}

\begin{IEEEproof}
Suppose this is not the case. Then there is some $r\in W_{N/2}$ for which
$r\notin K-K,$ so that%
\[
K\cap\left(  K+r\right)  =\emptyset.
\]
Let $K_{N/2}=K\cap W_{N/2}.$ Then $\#K_{N/2}>N/4,$ and the same is true of
$K\cap\left[  r+1,r+N/2\right]  .$ Since both $r+K_{N/2}$ and $K\cap\left[
r+1,r+N/2\right]  $ are subsets of $\left[  r+1,r+N/2\right]  ,$ are disjoint,
and each has more than $N/4$ elements, we have a contradiction.
\end{IEEEproof}

\begin{lemma} \label{lem:fracpartsumanddelta}
Let $\left|\fracpart{a}\right| < \delta$ and $\left|\fracpart{b}\right| < \delta$.  Then $\left|\fracpart{a + b}\right| < 2\delta$.
\end{lemma}
\begin{IEEEproof}
  If $\delta > \nicefrac{1}{4}$, the proof is trivial as $\left|\fracpart{a +
    b}\right| \leq \nicefrac{1}{2}$ for all $a,b \in \reals$.  If $\delta
  \leq \nicefrac{1}{4}$ then $\fracpart{a} + \fracpart{b} = \fracpart{a + b}$ and by the triangle
  inequality
\[
\left|\fracpart{a + b}\right| = \left|\fracpart{a} + \fracpart{b}\right| \leq \left|\fracpart{a}\right| + \left|\fracpart{b}\right| < 2\delta.
\] 
\end{IEEEproof}

\begin{lemma}
  \label{lem:moran2}Suppose $\left( \lambda_{N}\right) $ and $\left( \phi
    _{N}\right) $ are sequences with $\left( \lambda_{N},\phi_{N}\right) \in
  B$ and with%
\[
\frac{1}{N}\sum_{n=1}^{N}g\left(  \fracpart{  n\lambda_{N}+\phi_{N} }
\right)  \rightarrow 0,
\]
where $g\left(  x\right)  $ is continuous and even, and $g\left(  x\right)
\geq 0,$ with equality if and only if $\fracpart{x} = 0.$ Then
$N\lambda_{N}\rightarrow0$ and $\phi_{N} \rightarrow 0.$
\end{lemma}

\begin{IEEEproof}
For any $\delta>0,$ there exists $N_{0}$ such that if $N>N_{0}$ and
$K_{N}=\left\{  n\leq N;\left\vert \fracpart{  n\lambda_{N}+\phi_{N} }
\right\vert <\delta\right\}  $ then $\#K_{N}>3/4N.$ Choose $\delta<1/8.$ By
Lemma \ref{lem:moran},
\[
\left\{  1,2,\ldots,N/2\right\}  \subset K_{N}-K_{N}.
\]
Let $m,n\in K_{N}.$ Then%
\begin{align*}
\left\vert \fracpart{  n\lambda_{N}+\phi_{N} }  \right\vert  &  <\delta,\\
\left\vert \fracpart{  m\lambda_{N}+\phi_{N} }  \right\vert  &  <\delta
\end{align*}
and from Lemma \ref{lem:fracpartsumanddelta}
\[
\left\vert \fracpart{  \left(  m-n\right)  \lambda_{N} }  \right\vert
<2\delta.
\]
Thus, for $k\in\left\{  1,2,\ldots,N/2\right\}  ,$%
\[
\left\vert \fracpart{  k\lambda_{N} }  \right\vert <2\delta.
\]
In particular, $\left\vert \lambda_{N}\right\vert <2\delta<1/4,$ and so
$2\lambda_{N}=\fracpart{2\lambda_{N}}  .$ Putting $k=2$ above, we
therefore have $\left\vert 2\lambda_{N}\right\vert <2\delta,$ and so
$\left\vert \lambda_{N}\right\vert <\delta.$ Hence $\left\vert 4\lambda
_{N}\right\vert <4\delta<1/2,$ and so $4\lambda_{N}=\fracpart{  4\lambda
_{N} }  .$ Continuing in this way, we have $\left\vert 2^{r}\lambda
_{N}\right\vert <2\delta,$ for all $r$ such that $2^{r}\leq N/2.$ Let $r$ be
such that $2^{r}\leq N/2$ and $2^{r+1}>N/2.$ Then%
\[
\left\vert N\lambda_{N}\right\vert <\left\vert 2^{r+2}\lambda_{N}\right\vert
<8\delta.
\]
Since $\delta$ is arbitrary, it follows that $N\lambda_{N}\rightarrow0$ as
$N\rightarrow\infty.$

Let $n \in K_{N}$. From the above argument $n\lambda_{N}\rightarrow0$ and $\left\vert \fracpart{  n\lambda_{N}+\phi_{N} }  \right\vert < \delta$. Consequently $\left\vert \phi_{N}\right\vert <\delta$ and therefore $\phi_{N}\rightarrow0$.
\end{IEEEproof}

\begin{theorem}
\label{th:one}Let $\widehat{\lambda}_{N}$ and $\widehat{\phi}_{N}$ be given by
$\left(  \ref{eq:lsest}\right)  .$ Then $N\widehat{\lambda}_{N}\rightarrow0$
and $\widehat{\phi}_{N}\rightarrow0,$ almost surely as $N\rightarrow\infty.$
\end{theorem}

\begin{IEEEproof}
Let $A$ be the subset of the sample space $\Omega$ on which $\tau_{N}\left(
\widehat{\lambda}_{N},\widehat{\phi}_{N}\right)  \rightarrow\sigma^{2},$ and
let $\left(  \lambda_{N},\phi_{N}\right)  $ be $\left(  \widehat{\lambda}%
_{N},\widehat{\phi}_{N}\right)  $ at some point in $A.$ Note that $P\left(
A\right)  =1.$ Lemma \ref{lem:moran2}, applied to the function $g$ defined in
Lemma \ref{lem:one} shows that $N\lambda_{N}\rightarrow0$ and $\phi
_{N}\rightarrow0$. The proof follows.
\end{IEEEproof}

We are now in a position to complete the derivation of the asymptotic
properties.
\begin{IEEEproof} (Theorem \ref{thm:asymp_proof})
Strong consistency follows directly from Theorem \ref{th:one}. The proof
of the central limit theorem is along different lines from usual proofs, as the second
derivatives of  \eqref{eq:sumofsquaresfunction} do not exist everywhere in a
neighbourhood of the true values. Let%
\[
T_{N}\left(  \psi,\phi\right) =S_{N}\left(  \psi/N,\phi\right) =\frac{1}{N}\sum_{n=1}^{N}\fracpart{  X_{n}+n\psi/N+\phi }  ^{2}.
\]
In view of the strong consistency, it is only the local behaviour near
$\left(  \psi,\phi\right)  =\left(  0,0\right)  $ which is relevant. We shall
assume in what follows that $\left\vert \psi\right\vert +\left\vert
\phi\right\vert <1$. It is shown in Lemma \ref{lem:TNparts} in the appendix that
\begin{align}
\left[
\begin{array}
[c]{c}%
\frac{\partial T_{N}}{\partial\psi}\\
\frac{\partial T_{N}}{\partial\phi}%
\end{array}
\right] &= \frac{2}{N}\sum_{n=1}^{N}\left[
\begin{array}
[c]{c}%
n/N\\
1
\end{array}
\right]  X_{n} \nonumber \\
&  +2\left(  1-h  +o\left(  1\right) \right) \Abf  \left[
\begin{array}
[c]{c}%
\psi\\
\phi
\end{array}
\right]  \nonumber \\
&  +\left(  \left\vert \psi\right\vert +\left\vert \phi\right\vert
\right) O_P \left( N^{-1/2} \right)  . \label{eq:TNpart}
\end{align}
where $h = f_X \left(-1/2\right)$ and $\Abf$ is the $2 \times 2$ matrix
\[
\Abf = N^{-1}\sum_{n=1}^{N}\left[
\begin{array}
[c]{cc}%
\left(  n/N\right)  ^{2} & n/N\\
n/N & 1
\end{array}
\right].
\] 
Put%
\[
\left[
\begin{array}
[c]{c}%
\psi_{N}\\
\phi_{N}%
\end{array}
\right]  =-\frac{1}{N(1-h)} \Abf^{-1} \sum_{n=1}^{N}\left[
\begin{array}
[c]{c}%
n/N\\
1
\end{array}
\right]  X_{n}.
\]
Now%
\[
N^{-1/2}\sum_{n=1}^{N}\left[
\begin{array}
[c]{c}%
n/N\\
1
\end{array}
\right]  X_{n}%
\]
is asymptotically normal with mean $0$ and covariance matrix%
\[
\sigma^{2}\lim_{N\rightarrow\infty} \Abf  = \sigma^{2}\left[
\begin{array}
[c]{cc}%
1/3 & 1/2\\
1/2 & 1
\end{array}
\right]  ,
\]
where $\sigma^{2}=\operatorname{var}X_{n}.$ Thus $N^{1/2}\left[\psi_{N} \; \phi_{N} \right]^{\prime}$ is asymptotically normal with mean $0$ and covariance matrix%
\[
 \frac{\sigma^{2}}{\left(  1-h  \right)  ^{2}}\left[
\begin{array}
[c]{cc}%
1/3 & 1/2\\
1/2 & 1
\end{array}
\right]  ^{-1} = \frac{12\sigma^{2}}{\left(  1-h  \right)  ^{2}}\left[
\begin{array}
[c]{cc}%
1 & -1/2\\
-1/2 & 1/3
\end{array}
\right]  .
\]
Hence, since $\psi_{N}$ and $\phi_{N}$ are $O_{P}\left(  N^{-1/2}\right)  ,$
and%
\begin{align*}
\Abf^{-1} \left[
\begin{array}
[c]{c}%
\frac{\partial T_{N}}{\partial\psi}\\
\frac{\partial T_{N}}{\partial\phi}%
\end{array}
\right]
&  =-2\left(  1-h  \right)  \left(  \left[
\begin{array}
[c]{c}%
\psi_{N}\\
\phi_{N}%
\end{array}
\right]  -\left[
\begin{array}
[c]{c}%
\psi\\
\phi
\end{array}
\right]  \right)  \\
&  +2o\left(  1\right)  \left[
\begin{array}
[c]{c}%
\psi\\
\phi
\end{array}
\right]  + \left(  \left\vert \psi\right\vert +\left\vert
\phi\right\vert \right)  O_P \left( N^{-1/2}\right)  ,
\end{align*}
it follows that
\begin{equation}
\left[
\begin{array}
[c]{c}%
\frac{\partial T_{N}}{\partial\psi}\\
\frac{\partial T_{N}}{\partial\phi}%
\end{array}
\right]  =0\label{eq:part}%
\end{equation}
only if%
\[
\left[
\begin{array}
[c]{c}%
\psi\\
\phi
\end{array}
\right]  =\left[
\begin{array} 
[c]{c}%
\psi_{N}\\
\phi_{N}%
\end{array}
\right]  \left(  1+o_{P}\left(  1\right)  \right)  .
\]
The result follows, since $N\widehat{\lambda}_{N}=N\left(
  f_{0}-\widehat{f}_{N}\right)$ and
$\widehat{\phi}_{N}=\theta_{0}-\widehat{\theta}_{N}$ and since $\left(
  \ref{eq:part}\right) $ holds at $\left( \psi,\phi\right) =\left(
  N\widehat{\lambda}_{N},\widehat{\phi}_{N}\right) $ by Lemma \ref{lem:part} contained in the appendix.
\end{IEEEproof}

%%% Local Variables: 
%%% mode: latex
%%% TeX-master: "lsuwrapping.tex"
%%% End: 




%\section{Lower bounds for estimation}


\section{Least squares unwrapping and the nearest lattice point problem} \label{sec:LSPUandNP}

% \subsection{Lattice theory}\label{sec:lattice-theory}
In this section we describe methods to compute the LSPUE.  We find that the
computational problem can be transformed into a nearest lattice point problem
\cite{Agrell2002} in an $N-2$ dimensional lattice to be specified shortly.  
We first require some concepts from lattice theory.

The set $L$ is said to be a lattice, with \emph{generator} or \emph{basis} matrix  $\Bbf$ if \cite{SPLAG},
\[ 
  L = \{\pbf \in \reals^n \;|\; \pbf = \Bbf\wbf , \wbf \in \ints^n \}.
\]
Vectors and matrices are written in bold font and 
${}^{\prime}$ is used to denote transpose. A fundamental problem in lattice theory is
the \emph{nearest lattice point problem}. The nearest lattice point problem
is, given $\qbf\in\reals^n$ and some lattice $L$ whose lattice points lie
in~$\reals^n$, to find that lattice point $\pbf \in L$ for which the Euclidean
distance between $\qbf$ and $\pbf$ is minimised.  The notation
$\NP(\qbf, L)$ is used to denote the nearest point in $L$ to $\qbf$.  We
assume that, when two or more lattice points are of equal distance to $\qbf$, $\NP(\qbf, L)$
selects one of the lattice points in a systematic manner.

The nearest lattice point problem is known to be NP-hard under certain conditions when the lattice itself, or rather a basis thereof, is considered as an additional input parameter \cite{micciancio_hardness_2001, AjtaiShortestVecProbNPHard1998, Jalden2005_sphere_decoding_complexity}. Nevertheless, algorithms exist that can compute the nearest lattice point in reasonable time if the dimension is small
\cite{Agrell2002, Viterbo_sphere_decoder_1999, Pohst_sphere_decoder_1981}. One
such algorithm introduced by Pohst \cite{Pohst_sphere_decoder_1981} in 1981
was popularised in the signal processing and communications fields by Viterbo
and Boutros \cite{Viterbo_sphere_decoder_1999} and has since been called the
\emph{sphere decoder}. Approximate algorithms for computing the nearest point
have also been studied.  One example is Babai's nearest plane algorithm
\cite{Babai1986}, which requires $O(n^4)$ arithmetic operations in the worst
case.  For specific lattices where the generator matrix is known a priori,
many fast nearest point algorithms are known \cite{Conway1982FastQuantDec,
  Conway1986SoftDecLeechGolay, Clarkson1999:Anstar, McKilliam2008,
  McKilliam2008b, McKilliam2009CoxeterLattices}.

We now show how the LSPUE can be represented as a nearest lattice point
problem in a lattice determined by $N$. Define the $N$-dimensional vectors
$\mathbf{n}=\left[1 \; 2 \; \cdots \; N\right]^{\prime}$, $\mathbf{1}=\left[1 \; 1 \; \cdots \; 1\right]^{\prime}$, $\mathbf{y}=\left[Y_1 \; Y_2 \; \cdots \; Y_N\right]^{\prime}$, $\mathbf{x}=\left[X_1 \; X_2 \; \cdots \; X_N\right]^{\prime}$ and $\mathbf{u}=\left[U_1 \; U_2 \; \cdots \; U_N\right]^{\prime}$. The sum of squares function
\eqref{eq:sumofsquaresfunctionwithU} can be written in vector form as
\begin{equation}\label{eq:vectorsumofsquares}
  \left\|\ybf - f\nbf - \theta\onebf - \ubf \right\|^2.
\end{equation}
Define the matrix $\Mbf = [\nbf \; \onebf].$ The least squares estimator is then
\begin{equation}\label{eq_least_squares_unwrapping_x}
\left[ \begin{array}[c]{c} \widehat{f}_N \\ \widehat{\theta}_N \end{array} \right] = \arg\min_{(f,\theta)\in[-\nicefrac{1}{2},\nicefrac{1}{2})^2}\min_{\ubf \in \ints^N }\left\|\ybf - \Mbf\left[ \begin{array}[c]{c} f \\ \theta \end{array} \right] - \ubf \right\|^2.
\end{equation}

Given $\ubf$, the least squares estimators of $f$ and $\theta$ are obtained by
the usual linear regression formulae and are given by
\begin{equation} \label{eq_regf} \left[ \begin{array}[c]{c} \widehat{f}_N \\
      \widehat{\theta}_N \end{array} \right] = \Mbf^+ (\ybf - \ubf),
\end{equation}
where $\Mbf^+ = (\Mbf'\Mbf)^{-1}\Mbf'$.  Substituting (\ref{eq_regf}) into
(\ref{eq_least_squares_unwrapping_x}) and rearranging, the least squares
estimator of $\ubf$, given $f$ and $\theta$, is
\begin{equation} \label{eq_minu}
\hat{\ubf} = \arg\min_{\ubf\in\ints^{N}} {\|\Bbf(\ybf - \ubf)\|},
\end{equation}
where $\Bbf = \Ibf - \Mbf\Mbf^+$ and $\Ibf$ is the $N \times N$ identity
matrix.

Let $\Lambda$ be the lattice with generator matrix $\Bbf$.  The least squares
phase unwrapping is then the $\hat{\ubf}$ for which $\Bbf\hat{\ubf}$ is the
nearest lattice point in $\Lambda$ to $\Bbf\ybf$.  The least squares estimate
(\ref{eq_least_squares_unwrapping_x}) can be computed by first computing
$\NP(\Bbf\ybf, \Lambda)$, producing both $\Bbf\hat{\ubf}$ and $\hat{\ubf}$.
$\widehat{f}_N$ and $\widehat{\theta}_N$ can then be computed using
\eqref{eq_regf}.

The most difficult part of the procedure is finding the nearest lattice point.
One possible solution is to use the sphere decoder.  Unfortunately the sphere
decoder has worst case exponential complexity and therefore is only
computationally feasible when $N$ is small.  However, the lattice $\Lambda$ is
not random and has a generator matrix with significant structure.  Using this
structure we can construct a polynomial-time algorithm to compute the nearest
point.  %In this paper we are interested in finding the estimates of $\hat{f}$ and $\hat{\theta}$ so we will define these algorithms to output these estimates rather than the nearest lattice point $\Bbf\hat{\ubf}$.  If the nearest lattice point is desired it can be calculated from $\hat{f}$ and $\hat{\theta}$ as $\Bbf\round{\ybf - \hat{f}\nbf - \hat{\theta}\onebf}$.

Note that the nearest point is given by $\Bbf\hat{\ubf}$ where
\begin{equation}\label{eq:nearest_point_minimisation}
\hat{\ubf} = \arg\min_{\ubf\in\ints^N}\min_{(f, \theta)\in[-\nicefrac{1}{2},\nicefrac{1}{2})^2} \|\ybf - f\nbf - \theta\onebf - \ubf\|^2 .
\end{equation}
Fixing both $f$ and $\ubf$ and minimising with respect to $\theta$ we obtain
\begin{equation} \label{eq:minwrttheta} 
\theta = \frac{\onebf'(\ybf - f\nbf - \ubf)}{N}.
\end{equation}
Substituting this into \eqref{eq:nearest_point_minimisation} we find that
\begin{equation} \label{eq:min_An*_line}
\hat{\ubf} = \arg\min_{\ubf\in\ints^N}\min_{f\in[-\nicefrac{1}{2},\nicefrac{1}{2})} \|\Qbf\ybf - f\Qbf\nbf - \Qbf\ubf\|^2 ,
\end{equation}
where $\Qbf$ is the projection matrix
\begin{equation} \label{eq:Q}
\Qbf =  \left(\Ibf - \frac{\onebf \onebf'}{\onebf'\onebf}\right).
\end{equation}
The matrix $\Qbf$ is the generator matrix for a well studied lattice called
$A_{N-1}^*$ \cite{SPLAG}.  Numerous nearest point algorithms exist for the
lattice $A_N^*$ \cite{Clarkson1999:Anstar, Conway1982FastQuantDec,
  McKilliam2008a, McKilliam2008b}.  The fastest known algorithm was described
by McKilliam \emph{et al.} and requires $O(N)$ arithmetic operations
\cite{McKilliam2008b}.

Let $\zbf = \Qbf\ybf$, $\gbf = \Qbf\nbf$ and $\zetabf = \Qbf\ubf$.  Given $f$,
\eqref{eq:min_An*_line} is minimised when
\[
\zetabf = \NP(\zbf - f\gbf, A_{N-1}^*).
\]
Conversely, given $\zetabf$, \eqref{eq:min_An*_line} is minimised by putting
\[
f = \frac{\gbf'(\zbf - \zetabf)}{\gbf'\gbf}.
\]
If we define the set
\begin{equation}\label{eq:setS}
  S = \left\{ \NP(\zbf - f\gbf, A_{N-1}^*) \; ; \; f\in[-\nicefrac{1}{2},\nicefrac{1}{2}) \right\},
\end{equation}
then the nearest point is given by $\Bbf\hat{\ubf}$ where $\hat{\ubf}$ satisfies
\begin{equation} \label{eq:findbinS}
\Qbf\hat{\ubf} = \arg\min_{\zetabf \in S} h(\zetabf)
\end{equation}
and $h: \reals^N \mapsto \reals$ is defined by
\[
h(\zetabf) = \left\|\zbf - \frac{\gbf'(\zbf - \zetabf)}{\gbf'\gbf}\gbf - \zetabf \right\|^2.
\]
The set $S$ contains the lattice points in $A_{N-1}^*$ that are nearest to
some point on the line segment defined by $\zbf - f\gbf$ where
$f\in[-\nicefrac{1}{2},\nicefrac{1}{2})$.

Note that $\#S$ is finite.  We require an algorithm to find the lattice points
in $S$.  The nearest point can then be found by testing each lattice point and
returning the minimiser according to (\ref{eq:findbinS}). Computing $S$
directly proves to be difficult, and, in fact, it is easier to compute a
superset of $S$.  To show this we require some properties of the lattice
$A_{N-1}^*$.  The generator matrix for $A_{N-1}^*$ is the matrix $\Qbf$
defined in \eqref{eq:Q}, which is the projection matrix into the hyperplane
orthogonal to $\onebf$.  Let $H$ denote this hyperplane.  Then $A_{N-1}^*$
consists of the lattice points in $\ints^N$ projected orthogonally onto $H$.
That is,
\[
A_{N-1}^* = \left\{ \Qbf\wbf \mid \wbf \in \ints^N \right\}.
\]
%It is easy to see that
%\begin{equation} \label{eq:VorAn*subsetvorZn}
%\vor(A_{N-1}^*) \cap H \subset \vor(\ints^N)
%\end{equation}

Another way of representing $A_{N-1}^*$ is as a union of $N$ translates of its dual lattice $A_{N-1} = \ints^{N} \cap H$ \cite{SPLAG}.  That is,
\begin{align}
  A_{N-1}^* &= \bigcup_{i=0}^{n-1} \left( [i] + A_{N-1} \right) \nonumber \\
  					&= \bigcup_{i=0}^{n-1} \left( [i] + \ints^N \right) \bigcap H \\
            &\subset \bigcup_{i=0}^{n-1} \left( [i] + \ints^N \right),\label{eq:An*asglues}
\end{align}
where the $[i]$ are known as \emph{glue vectors} and are defined in
this case as
\begin{equation} \label{eq:Anglues}
  [i] = \frac{1}{N} \big[ \underbrace{i, \dots, i}_{\text{$j$ times}},
        \underbrace{-j, \dots, -j}_{\text{$i$ times}}
  \big]
\end{equation}
for $i = 0,1, \dots, N-1$ with $i+j = N$ \cite[pp. 109]{SPLAG}.

There exists a fast algorithm to find the points in the set
\[
K(i) = \left\{ \NP(\zbf - f\gbf, [i] + \ints^N) \mid f \in [-\nicefrac{1}{2},\nicefrac{1}{2}) \right\}
\]
due to Ryan \emph{et al.} \cite{Ryan2007, McKilliam2007}.  We use this algorithm for each $i = 0,1,\dots,N-1$. Let
\begin{equation} \label{eq:S+}
S^+ = \bigcup_{i=0}^{n-1} K(i).
\end{equation}

The proof of the following lemma is not difficult and is omitted.

\begin{lemma} \label{lem:minSS+equal}
\[
\Qbf\hat{\ubf} = \arg\min_{\zetabf \in S} h(\zetabf) = \arg\min_{\zetabf \in S^+} h(\zetabf).
\]
\end{lemma}

It is thus sufficient to use the set $S^+$ rather than $S$ to find the nearest lattice point.  Since $S^+$ is the union of the $K(i)$,
\[
\Qbf\hat{\ubf} = \arg\min_{\zetabf \in K(i)}\min_{i=0,1,\dots,N-1} h(\zetabf).
\]
For each $i$ the minimisation can be computed using the algorithm in \cite{Ryan2007, McKilliam2007}.  The number of operations required is $O(\#S^+\log{N})$.

\begin{lemma} \label{lem:sizeK=O(N^2)}
For $i=0,1,\dots,N$
\[
\#K(i) = O(N^2).
\]
\end{lemma}
\begin{IEEEproof}
The `first' lattice point in $K(i)$ corresponds to $f = -\nicefrac{1}{2}$, and is given by
\[
\vbf = \round{\zbf + \frac{\gbf}{2} - [i]} + [i],
\]
while the `last' lattice point in $K(i)$ corresponds to $f = \nicefrac{1}{2}$, and is given by
\[
\wbf = \round{\zbf - \frac{\gbf}{2} - [i]} + [i].
\]
Consecutive elements in $K(i)$ satisfy the equation
\[
\vbf_{\text{next}} = \vbf_{\text{previous}} + \sign{g_j}\ebf_j 
\]
for some $j$, where $\ebf_j$ is a vector conistsing of zeros apart from a one in the $j$th position and $\sign{g_j}$ is $1$ if $g_j > 0$, $-1$ if $g_j < 0$ and $0$ if $g_j = 0$.  Thus
\[
\#K(i) = \sum_{n=1}^N |v_n - w_n| \leq \sum_{n=1}^N |w_n| + |v_n|.
\]
Since $z_n$ and $[i]_n \in [-\nicefrac{1}{2}, \nicefrac{1}{2})$, and $|g_n| \leq N$, $|v_n|,|w_n| \leq N+2$ for all $n=1,2,\dots,N$.  Hence
\[
\#K(i) \leq \sum_{n=1}^N{ \left(2N + 4\right) } = O(N^2).
\] 
\end{IEEEproof}

As a consequence of the lemma, $\#S^+ = O(N^3)$ and the algorithm requires $O(N^3\log{N})$ arithmetic operations. 


% \begin{algorithm} \label{alg:ON^3logN}
% \SetAlCapFnt{\small}
% \SetAlTitleFnt{}
% \dontprintsemicolon
% $\zbf = \Qbf\ybf$ \;
% $\gbf = \Qbf\nbf$ \;
% $D = \infty$ \;
% \For{$i = 0$ \emph{\textbf{to}} $n-1$}{ 
% 	$\vbf = \round{\zbf - [i]} + [i]$\;
% 	\For{$t = 1$ \emph{\textbf{to}} $n$ \nllabel{alg:alg_for_glue}}{
% 		\If{$g_t \neq 0$}{
% 			$f^* = \frac{z_t - v_t + 0.5\sign{g_t}}{g_t}$ \;
% 			$\operatorname{AddToSortedList}(f^*, t)$ \;
% 		}
% 	}
% 	$\alpha = \gbf'(\vbf - \zbf)$ \;
% 	$\beta = \|\zbf - \vbf\|^2$ \;
% 	$\gamma = \|\gbf\|^2$ \;
% 	$f = \frac{\gbf'(\zbf - \vbf)}{\gbf'\gbf}$ \;
% \While{$f < 1$ \nllabel{alg:whilef<1}}{
% 	$h =  \beta + 2f\alpha + f^2\gamma$ \;\
% 	\If{$h < D$ \nllabel{snpe:alg:alg_ifL}}{
% 		$D = h$ \;
% 		$\hat{f} = f$ \;
% 	}
% 	$t = \operatorname{RemoveFirstFromSortedList}$ \;
% 	$\alpha = \alpha - |g_{t}|$ \;
% 	$\beta = \beta + 2\sign{g_{t}} \left( z_{t} - v_{t} \right) + 1$\;
% 	$f = f + \frac{|g_{t}|}{\|\gbf\|^2}$ \;
% 	$v_{t} = v_{t} - \sign{g_t}$\; 
% 	$f^* = \frac{z_t - v_t + 0.5\sign{g_t}}{g_t}$ \;
% 	$\operatorname{AddToSortedList}(f^*, t)$ \;
% 	}
% }
% $(\zetabf, \hat{\ubf}) = \NP(\zbf - \hat{f}\gbf, A_{N-1}^*)$ \;
% \Return{$(\Bbf\hat{\ubf}, \hat{\ubf})$ \nllabel{snpe:alg:alg_returnT}}
% \caption{$O(N^3\log{N})$ algorithm to find the nearest lattice point}
% \end{algorithm}

%%% Local Variables: 
%%% mode: latex
%%% TeX-master: "lsuwrapping.tex"
%%% End:



\section{Simulations} \label{sec:simulations} 

We have compared the performance of five 
estimators: the periodogram estimator
\cite{Rife1974}; the LSPUE; the parabolic, smoothed central finite difference
estimator (PSCFD) \cite{Lovell1991}; Kay's window estimator \cite{Kay1989};
and the Quinn-Fernandes estimator \cite{Quinn_Fernandes_1991,
  Quinn_recent_advances_in_freq_est_2008}.  Those estimators based on phase
unwrapping are the LSPUE, Kay's estimator and the PSCFD estimator. Five
simulations were run with $N=16$, $N=64$ , $N=256$, $N=512$ and $N=1024$ (Figures \ref{plot:MSEvSNRn=16} to \ref{plot:MSEvSNRn=1024} respectively), each with SNR  varied between \unit[-20]{dB} and \unit[20]{dB},  and 1000 trials were run for each SNR value.  The value of $(f_0, \theta_0)$ was varied uniformly
in the range $[-\nicefrac{1}{2},\nicefrac{1}{2})^2$.  The distribution of the
$s_n$ was assumed to be complex i.i.d. and Gaussian with variance $\sigma^2$. 
This gave an SNR of $10 \log_{10} \left( \nicefrac{A^2}{2\sigma^2} \right)$ dB.

Standard behaviour was observed for the `nonlinear' estimators.  The mean
square error (MSE) was large below a particular threshold SNR.  Above that
threshold, the estimators appeared to converge to the Cramer-Rao lower bound
(CRB) \cite{Rife1974} depicted by the dashed line.  It is clear that the
periodogram estimator produces the most statistically accurate results, with
the LSPUE and Quinn-Fernandes estimators only marginally worse.  Kay's
estimator and the PSCFD perform comparatively poorly, particularly for large
$N$.  Kay's estimator performs particularly poorly as it fails to correctly
estimate $f_0$ when it is near $0.5$, i.e., when $f_0$ is near the branch cut
on the unit circle \cite{Clarkson1999}.  As noted elsewhere
\cite{Clarkson1994AnalysisKaysVariance}, the performance of Kay's estimator is
improved if $f_0$ is bounded away from $0.5$.

The dash-dotted line is the asymptotic variance of the LSPUE derived in
Theorem \ref{thm:asymp_proof}.  It can be seen that, provided the SNR is high
enough to avoid the threshold effect, the performance of the LSPUE closely
agrees with the asymptotic results.  Note that the asymptotic variance of the LSPUE is larger than the CRB.  This performance loss can be overcome by using a numerical optimisation procedure, such as Newton's method, starting at the estimate given by the LSPUE.  In order to show the correctness of our asymptotic theory we have not displayed these results here.

Table \ref{tab_computation_time} shows the computation time for $10^5$ trials of each estimator for $N=16,64,256,512,1024$. The computer used is a \unit[2.13]{Ghz} Intel Core2. As expected the LSPUE is significantly slower than other estimators. The computational complexity of our LSPUE algorithm (Section \ref{sec:LSPUandNP}) has order $O(N^3\log{N})$ whereas the other estimators have complexity $O(N)$ or $O(N\log{N})$.  For this reason, the periodogram or Quinn-Fernandes estimators are to be preferred in practice. Nevertheless, it may be that significantly faster algorithms exist to compute the LSPUE.

\begin{table}[h]
\centering
\caption{Computation time in seconds for $10^5$ trials}
\begin{tabular}{lrrrrr}
Algorithm & \multicolumn{1}{l}{n=16} & \multicolumn{1}{l}{n=64} & \multicolumn{1}{l}{n=256} & \multicolumn{1}{l}{n=512} & \multicolumn{1}{l}{n=1024} \\ \toprule
Kay  & 0.156 & 0.625 & 2.406 & 4.813 & 9.641\\ 
PSCFD  & 0.141 & 0.547 & 2.094 & 4.187 & 8.375\\ 
Quinn-Fernandes  & 0.438 & 1.297 & 4.469 & 8.985 & 21.141\\ 
Periodogram & 0.437 & 1.656 & 7.578 & 17.828 & 49.157\\
LSPUE & 5.578 & 346.2 & $>10^4$ & $>10^5$ & $>10^6$ \\ \bottomrule
\end{tabular}
\label{tab_computation_time}
\end{table}

\begin{figure}[htbp]
	\centering
		\includegraphics[width=\linewidth]{code/data/plot-1.mps}
		\caption{MSE in frequency versus SNR when $N=16$.}
		\label{plot:MSEvSNRn=16}
\end{figure}

\begin{figure}[htbp]
	\centering
		\includegraphics[width=\linewidth]{code/data/plot-2.mps}
		\caption{MSE in frequency versus SNR when $N=64$}
		\label{plot:MSEvSNRn=64}
\end{figure}

\begin{figure}[htbp]
	\centering
		\includegraphics[width=\linewidth]{code/data/plot-3.mps}
		\caption{MSE in frequency versus SNR when $N=256$.}
		\label{plot:MSEvSNRn=256}
\end{figure}

\begin{figure}[htbp]
	\centering
		\includegraphics[width=\linewidth]{code/data/plot-4.mps}
		\caption{MSE in frequency versus SNR when $N=512$.}
		\label{plot:MSEvSNRn=512}
\end{figure}

\begin{figure}[htbp]
	\centering
		\includegraphics[width=\linewidth]{code/data/plot-5.mps}
		\caption{MSE in frequency versus SNR when $N=1024$.}
		\label{plot:MSEvSNRn=1024}
\end{figure}

%\subsection{Computational Trials}


\section{Conclusion}\label{sec:conclusion}

We have discussed single frequency estimation via least squares phase
unwrapping. This estimator has been shown to be strongly consistent and its
central limit theorem derived. The problem of computing the least squares
phase unwrapping has been demonstrated to be related to a problem in
algorithmic number theory known as the nearest lattice point problem. We have
derived an algorithm that computes the least squares estimate in
$O(N^3\log{N})$ arithmetic operations where $N$ is the sample size.  The
complexity is high when compared with other single frequency estimators and
arises from the need to solve the nearest lattice point problem in the lattice
$\Lambda$ derived in Section \ref{sec:LSPUandNP}.  One possible algorithm has
been described here.  However, it may be that much faster nearest point
algorithms exist for this specific lattice.

We have compared the performance of the LSPUE and the periodogram estimator
\cite{Rife1974} by Monte Carlo simulation.  It was found that the LSPUE is
marginally less accurate than the periodogram estimator and significantly more
accurate than other estimators based on phase unwrapping.  The simulations
agree with the theoretical central limit theorem derived in Section
\ref{sec:asymptotic_properties}.


%\bibliographystyle{IEEEbib}
\small
\bibliography{bib}


\appendix

\begin{lemma} \label{lem:zero_mean_indepent_sum_bound}
  Let $Z_1, \dots, Z_N$ be independent, zero-mean random variables with $|Z_j|
  \leq 1$.  Then, for any integer $\beta > 0$, as $N \rightarrow \infty$,
\begin{equation*}
  S = E(Z_1 + \dots + Z_N)^{2 \beta} = O(N^\beta).
\end{equation*}
\end{lemma}

\begin{IEEEproof}
We can write $S$ according to the multinomial expansion
\begin{equation}
  S = \sum_{k_1 + \dots + k_N = 2 \beta} {2 \beta \choose k_1, \dots, k_N} \prod_{j=1}^N E(Z_j^{k_j})
  	\label{S=multinomial}
\end{equation}
where we make use of the multinomial coefficients
\begin{equation*}
  {2 \beta \choose k_1, \dots, k_N} = \frac{(2\beta)!}{k_1! \cdots k_N!}.
\end{equation*}

Now, because the $Z_j$ have zero mean, the product in~(\ref{S=multinomial}) is
zero if any $k_j = 1$.  Accordingly,  we define the set
\begin{equation*}
  \mathcal{K} = \bigg\{\kbf \in {\mathbb Z}^n \mid k_j \geq 0, k_j \neq 1, \sum_{j=1}^N k_j = 2 \beta\bigg\}
  \subset [0, 2 \beta]^N.
\end{equation*}
In view of the fact that the $Z_j$ are bounded with $|Z_j| \leq 1$, we then
have
\begin{equation}
  |S| \leq \sum_{\kbf \in \mathcal{K}} {2 \beta \choose \kbf}  \label{absSbound}
\end{equation}

Let $c(\kbf)$ be the number of non-zero elements in the vector $\kbf$.  Since
$k_j \neq 1$ and the $k_j$ sum to $2 \beta$, it follows that $c(\kbf) \leq
\beta$ for all $\kbf \in \mathcal{K}$.  Clearly, in addition, $c(\kbf) \geq 1$,
and so from (\ref{absSbound}) we have
\begin{equation*}
  |S| \leq \sum_{d=1}^{\beta} \sum_{\substack{\kbf \in \mathcal{K} \\ c(\kbf) = d}} {2 \beta \choose \kbf}
	\leq \sum_{d=1}^{\beta} \sum_{\substack{\kbf \in \mathcal{K} \\ c(\kbf) = d}} \frac{(2 \beta)!}{2^d}
	\leq \sum_{d=1}^{\beta} \sum_{\substack{\kbf \in [0, 2 \beta]^N \\ c(\kbf) = d}} \frac{(2 \beta)!}{2^d}
\end{equation*}

Consider the number of integer vectors $\kbf$ that satisfy the conditions of
the innermost sum, \emph{i.e.}, that $\kbf \in [0, 2\beta]^N$ and $c(\kbf) =
d$.  There are ${N \choose d}$ ways of selecting the indices of non-zero $k_j$
so that $c(\kbf) = d$ and then $2 \beta$ possibilities for the value of each
such non-zero $k_j$.  Therefore, there are $(2 \beta)^d {N \choose d}$
possible vectors in total.  Hence,
\begin{align*}
  |S| \leq (2 \beta)! \sum_{d=1}^{\beta} {N \choose d} \beta^d &\leq (2 \beta)! \sum_{d=1}^{\beta} (N \beta)^d \\ &\leq (2 \beta)! \beta (N \beta)^{\beta} = O(N^\beta).
\end{align*}
\end{IEEEproof}

\begin{lemma} \label{lem:TNparts}
Assume that $\left\vert \psi\right\vert +\left\vert\phi\right\vert <1$.  Then \eqref{eq:TNpart} holds.
\end{lemma}
\begin{IEEEproof}
\begin{align*}
\left[
\begin{array}
[c]{c}%
\frac{\partial T_{N}}{\partial\psi}\\
\frac{\partial T_{N}}{\partial\phi}%
\end{array}
\right]   &  =\frac{2}{N}\sum_{n=1}^{N}\left[
\begin{array}
[c]{c}%
n/N\\
1
\end{array}
\right]  \left(  X_{n}+n\psi/N+\phi-I_{n,N}\right)  \\
&  =\frac{2}{N}\sum_{n=1}^{N}\left[
\begin{array}
[c]{c}%
n/N\\
1
\end{array}
\right]  \left[  X_{n}+n\psi/N+\phi \right. \\ 
&\left. \hspace{2.7cm} -E\left(  I_{n,N}\right)  +E\left(
I_{n,N}\right)  -I_{n,N}\right]
\end{align*}
where, since $\forall n$ $\left\vert n\psi/N+\phi\right\vert <1,$%
\[
I_{n,N}=\left\{
\begin{array}
[c]{ccc}%
1 & ; & X_{n}+n\psi/N+\phi\geq \nicefrac{1}{2}\\
-1 & ; & X_{n}+n\psi/N+\phi<-\nicefrac{1}{2}\\
0 & ; & \text{otherwise.}%
\end{array}
\right.
\]
Now, if $n\psi/N+\phi>0,$%
\begin{align}
E\left(  I_{n,N}\right)   &  =P\left(  X_{n}+n\psi/N+\phi\geq \nicefrac{1}{2} \right)
\label{eq:one1}\\
&  =\int_{\nicefrac{1}{2}-\left(  n\psi/N+\phi\right)  }^{\nicefrac{1}{2}}f_X \left(  x\right)
dx 
 =\left(  n\psi/N+\phi\right)  f_X \left(  \xi_{n,N}\right) \nonumber
\end{align}
where $\nicefrac{1}{2}-\left(  n\psi/N+\phi\right)  \leq\xi_{n,N} < \nicefrac{1}{2},$ while, if
$n\psi/N+\phi<0,$%
\begin{align}
E\left(  I_{n,N}\right)   &  =-P\left(  X_{n}+n\psi/N+\phi<-1/2\right)
\label{eq:one2}\\
&  =\left(  n\psi/N+\phi\right)  f_X \left(  \xi_{n,N}\right)  ,\nonumber
\end{align}
where $-\nicefrac{1}{2}\leq\xi_{n,N}\leq-\nicefrac{1}{2}-\left(  n\psi/N+\phi\right)  .$ Thus, if
$n\psi/N+\phi>0,$
\[
\operatorname{var}I_{n,N} =E\left(  I_{n,N}\right)  -\left[  E\left(
I_{n,N}\right)  \right]  ^{2}  <\left(  n\psi/N+\phi\right)  f_X \left(  \xi_{n,N}\right)  ,
\]
while, if $n\psi/N+\phi<0,$%
\[
\operatorname{var}I_{n,N}<-\left(  n\psi/N+\phi\right)  f_X \left(  \xi
_{n,N}\right)  .
\]
Hence%
\begin{align*}
\operatorname{var}N^{-\nicefrac{1}{2}}\sum_{n=1}^{N}\left[  I_{n,N}-E\left(
I_{n,N}\right)  \right] &< \frac{4}{N}\sum_{n=1}^{N}\left\vert n\psi/N+\phi\right\vert f_X \left(
\xi_{n,N}\right)  \\
&  <\frac{4}{N}\left(  \left\vert \psi\right\vert +\left\vert \phi\right\vert
\right)  \sum_{n=1}^{N}f_X \left(  \xi_{n,N}\right)  .
\end{align*}
Similarly,%
\begin{align*}
&\operatorname{var}N^{-\nicefrac{1}{2}}\sum_{n=1}^{N}\frac{n}{N}\left[  I_{n,N}-E\left(
I_{n,N}\right)  \right] \\
&  < \frac{4}{N}\sum_{n=1}^{N}\left(  \frac{n}{N}\right)  ^{2}\left\vert
n\psi/N+\phi\right\vert f_X \left(  \xi_{n,N}\right)  \\
&  < \frac{4}{N}\left(  \left\vert \psi\right\vert +\left\vert \phi\right\vert
\right)  \sum_{n=1}^{N}f_X \left(  \xi_{n,N}\right)  .
\end{align*}
Since $f_X$ is symmetric and unimodal, with mode at $0$, there exists a unique
$\xi$, for which $f_X \left(  x\right)  <1$ when $\left\vert x\right\vert >\xi.$
Thus, as long as $\left\vert \psi\right\vert +\left\vert \phi\right\vert
<\frac{1}{2} - \xi,$%
\[
\frac{4}{N}\left(  \left\vert \psi\right\vert +\left\vert \phi\right\vert
\right)  \sum_{n=1}^{N}f_X \left(  \xi_{n,N}\right)  <4\left(  \left\vert
\psi\right\vert +\left\vert \phi\right\vert \right)
\]
and so
\[
N^{-1}\sum_{n=1}^{N}\left[
\begin{array}[c]{c}%
n/N \\
1
\end{array}
\right]  \left[  I_{n,N}-E\left(  I_{n,N}\right)  \right]  \\
 = \left(  \left\vert \psi\right\vert +\left\vert \phi\right\vert
\right) O_P \left( N^{-1/2} \right) .
\]
But, using $\left(  \ref{eq:one1}\right)  $ and $\left(  \ref{eq:one2}\right)  ,$
we obtain%
\begin{align*}
&  \frac{2}{N}\sum_{n=1}^{N}\left[
\begin{array}
[c]{c}%
n/N\\
1
\end{array}
\right]  \left[  X_{n}+n\psi/N+\phi-E\left(  I_{n,N}\right)  \right]  \\
&  =\frac{2}{N}\sum_{n=1}^{N}\left[
\begin{array}
[c]{c}%
n/N\\
1
\end{array}
\right]  \left[  X_{n}+n\psi/N+\phi-\left(  n\psi/N+\phi\right)  f_X \left(
\xi_{n,N}\right)  \right]  \\
&  =\frac{2}{N}\sum_{n=1}^{N}\left[
\begin{array}
[c]{c}%
n/N\\
1
\end{array}
\right]  \left[  X_{n}+\left(  n\psi/N+\phi\right)  \left(  1-f_X \left(
\xi_{n,N}\right)  \right)  \right]  ,
\end{align*}
and so%
\begin{align*}
\left[
\begin{array}
[c]{c}%
\frac{\partial T_{N}}{\partial\psi}\\
\frac{\partial T_{N}}{\partial\phi}%
\end{array}
\right]   &  = \frac{2}{N}\sum_{n=1}^{N}\left[
\begin{array}[c]{c}%
n/N\\
1
\end{array}
\right]  X_{n} \\
&-\frac{2}{N}\sum_{n=1}^{N}\left[
\begin{array}
[c]{c}%
n/N\\
1
\end{array}
\right]  \left[  I_{n,N}-E\left(  I_{n,N}\right)  \right] \\
&+ \frac{2}{N}\sum_{n=1}^{N}\left(  1-f_X \left(  \xi_{n,N}\right)
\right)  \left[
\begin{array}[c]{cc}%
\left(  n/N\right)  ^{2} & n/N\\
n/N & 1
\end{array}
\right]  \left[
\begin{array}
[c]{c}%
\psi\\
\phi
\end{array}
\right]  \\
&  =\frac{2}{N}\sum_{n=1}^{N}\left[
\begin{array}
[c]{c}%
n/N\\
1
\end{array}
\right]  X_{n} +\left(  \left\vert \psi\right\vert +\left\vert \phi\right\vert
\right) O_P \left( N^{-1/2} \right) \\
&\hspace{-0.9cm}+\frac{2}{N}\sum_{n=1}^{N}\left(  1-f_X \left( -1/2\right)  +o\left(  1\right) \right)  
\left[ \begin{array}[c]{cc}%
\left(  n/N\right)  ^{2} & n/N\\
n/N & 1
\end{array} \right]  
\left[ \begin{array}[c]{c}%
\psi\\
\phi
\end{array}
\right]
\end{align*}
where, because $|n\psi/N+\phi| \rightarrow 0$ as $N\rightarrow\infty$, $\xi_{n,N}$ approaches either $\nicefrac{1}{2}$ or $-\nicefrac{1}{2}$ for all $n$ and therefore $f_X\left(\xi_{n,N}\right) \rightarrow f_X\left( -1/2 \right)$.
\end{IEEEproof}

\begin{lemma}
\label{lem:part}The partial derivatives of \eqref{eq:sumofsquaresfunction} are
zero with probability 1 at $\left(  \widehat{f}_{N},\widehat{\theta}_{N}\right)  ,$ the minimiser
of \eqref{eq:sumofsquaresfunction} over $B$.
\end{lemma}

\begin{IEEEproof}
Let
\[
Z\left(  f,\theta\right)  =\sum_{n=1}^{N}\left(  Y_{n}-nf-\theta-\left\lfloor
Y_{n}-n\widehat{f}_{N}-\widehat{\theta}_{N}\right\rceil \right)  ^{2}.
\]
Let $\left(  f^{\prime},\theta^{\prime}\right)  $ be the minimiser of
$Z\left(  f,\theta\right)$. Observe that $Z$ is quadratic in
$\left( f,\theta \right)$ and so the partial derivatives of $Z$ at the unique minimiser 
$\left(  f^{\prime},\theta^{\prime}\right)$ are $0$. Now%
\[
Z\left(  f^{\prime},\theta^{\prime}\right)  \leq Z\left(  \widehat{f}%
_{N},\widehat{\theta}_{N}\right)  =SS\left(  \widehat{f}_{N},\widehat{\theta
}_{N}\right)  \leq SS\left(  f^{\prime},\theta^{\prime}\right)  .
\]
Thus $Z\left(  f^{\prime},\theta^{\prime}\right)  \leq SS\left(  f^{\prime}%
,\theta^{\prime}\right)$. However, $\forall n$
\[
\left\vert Y_{n}-nf^{\prime}-\theta^{\prime}-\round{ Y_{n}-n\widehat
{f}_{N}-\widehat{\theta}_{N} } \right\vert \geq\fracpart{
Y_{n}-nf^{\prime}-\theta^{\prime} }  .
\]
Hence $Z\left(  f^{\prime},\theta^{\prime}\right)  \geq SS\left(  f^{\prime},\theta^{\prime}\right)$, and so $Z\left(  f^{\prime},\theta^{\prime}\right)  =SS\left(  f^{\prime},\theta^{\prime}\right)$ and $Z\left(  f^{\prime},\theta^{\prime}\right)  =Z\left(  \widehat{f}_{N},\widehat{\theta}_{N}\right)$. Consequently $f^{\prime}=\widehat{f}_{N}$ and $\theta^{\prime}=\widehat{\theta}_{N}.$ The partial derivatives of $Z$ and $SS$ are thus identical whenever the latter's exist, and since the derivatives of $SS$ exist everywhere except on $\cup_{n}\left\{  \left(  f,\theta\right)  ; \fracpart{  Y_{n}-nf-\theta } =-1/2\right\}$, which has probability zero, the result follows\footnote{It is possible to remove the `with probability zero' statement by appealing to some concepts in lattice theory that we describe in Section \ref{sec:LSPUandNP}.  Due to space restrictions we have not included this result.  Theorem \ref{thm:asymp_proof} holds in any case.}.
 % We will show in Lemma \ref{lem:resinvoran} in the appendix, that $\left|\left\{Y_{n}-n\widehat{f}_{N}-\widehat{\theta}_{N}\right\}\right| \leq \nicefrac{1}{2} - \nicefrac{1}{2N}$ and therefore $SS$ is differentiable at $(\widehat{f}_{N}, \widehat{\theta}_{N})$ and the partial derivatives are zero.
\end{IEEEproof}


% Here we will prove a result required in Lemma \ref{lem:part}.  The result requires some concepts from lattice theory that were discussed in Section \ref{sec:LSPUandNP}. We firstly require the additional concept of \emph{relevant vectors}.  Consider the Voronoi region of a lattice $L$.  The faces of $\vor(L)$ lie in hyperplanes that are on the midway point between the lines connecting nearby lattice points.  The set of vectors that define the faces are called the \emph{Voronoi relevant vectors} or simply \emph{relevant vectors}.

% \begin{remark} \label{rem:relvec}
% Let $\rbf$ be a relevant vector in the lattice $L$.  If $\ybf \in \vor(L)$ then
% \[
% \ybf\cdot\rbf \leq \frac{\rbf\cdot\rbf}{2}.
% \]
% \end{remark}

% \begin{remark} \label{An*relvecs}
% The relevant vectors for the lattice $A_{N-1}^*$ are given by the vectors $\Qbf\ubf$ where
% \[
% \ubf = \sum_{i \in P}{\ebf_i}
% \]
% and where $P \subset \{1, 2, \dots, N\}$, $\ebf_i$ denotes a vector of zeros with a $1$ in the $i$th position and $\Qbf$ is given in (\ref{eq:Q}).
% \end{remark}
% A proof of Remark \ref{An*relvecs} is given in \cite{Clarkson1999:Anstar}.

% \begin{lemma} \label{lem:resinvoran}
% Let $\left(  \widehat{f}_{N},\widehat{\theta}_{N}\right)$ be the minimiser
% of \eqref{eq:sumofsquaresfunction} over $B$.  Then
% $\left|\left\{ Y_n - \widehat{f}_{N}n - \widehat{\theta}_{N} \right\}\right| \leq \nicefrac{1}{2} - \nicefrac{1}{2N}$ for all $n=1,2,\dots,N$.
% \end{lemma}
% \begin{IEEEproof}
% We have shown in Section \ref{sec:LSPUandNP} that
% \[
% \Qbf(\ybf - \widehat{f}_N \nbf - \widehat{\ubf}) \in \vor(A_{N-1}^*)
% \]
% where $\widehat{\ubf} =  \round{\ybf - \widehat{f}_N \nbf -  \widehat{\theta}_N\onebf}$. Now from \eqref{eq:minwrttheta} it follows that
% \[
%  \widehat{\theta}_N = \frac{\onebf'(\ybf - \widehat{f}_N \nbf - \widehat{\ubf})}{N}
% \] 
% and therefore
% \begin{align}
% \Qbf(\ybf - \widehat{f}_N \nbf - \ubf) &= \ybf - \widehat{f}_N \nbf -\widehat{\ubf} - \frac{\onebf'(\ybf - \widehat{f}_N \nbf - \widehat{\ubf})}{N}\onebf \nonumber \\
% &= \left\{ \ybf - \widehat{f}_N \nbf -  \widehat{\theta}_N\onebf \right\}  \in \vor(A_{N-1}^*) \in \label{eq:inVorAn*}
% \end{align}
% where we define $\{\cdot\}$ to operate on vectors by taking the fractional part of each element.  Let $\zbf = \left\{ \ybf - \widehat{f}_N \nbf -  \widehat{\theta}_N\onebf \right\}$. Assume that the lemma is false then $|z_n| > \nicefrac{1}{2} - \nicefrac{1}{2N}$ for some $n$.  If $z_n > \nicefrac{1}{2} - \nicefrac{1}{2N}$ then
% \[
% \Qbf\ebf_n \cdot \zbf = z_n - \bar{z} = z_n > \nicefrac{1}{2} - \nicefrac{1}{2N} = \frac{ \|\Qbf\ebf_n\|^2}{2}.
% \]
% This violates that $\zbf \in \vor{A_{N-1}^*}$ as $\Qbf\ebf_n$ is a relvant vector in $A_{N-1}^*$.  The case when  $z_n < -\nicefrac{1}{2} + \nicefrac{1}{2N}$ follows similarly.
% \end{IEEEproof}


\end{document}

%%% Local Variables: 
%%% mode: latex
%%% TeX-master: t
%%% End: 
