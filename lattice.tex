\section{Least squares unwrapping and the nearest lattice point problem} \label{sec:LSPUandNP}

% \subsection{Lattice theory}\label{sec:lattice-theory}
In this section we describe methods to compute the LSPUE.  We find that the
computational problem can be transformed into a nearest lattice point problem
\cite{Agrell2002} in an $N-2$ dimensional lattice to be specified shortly.  
We first require some concepts from lattice theory.

The set $L$ is said to be a lattice, with \emph{generator} or \emph{basis} matrix  $\Bbf$ if \cite{SPLAG},
\[ 
  L = \{\pbf \in \reals^n \;|\; \pbf = \Bbf\wbf , \wbf \in \ints^n \}.
\]
Vectors and matrices are written in bold font and 
${}^{\prime}$ is used to denote transpose. A fundamental problem in lattice theory is
the \emph{nearest lattice point problem}. The nearest lattice point problem
is, given $\qbf\in\reals^n$ and some lattice $L$ whose lattice points lie
in~$\reals^n$, to find that lattice point $\pbf \in L$ for which the Euclidean
distance between $\qbf$ and $\pbf$ is minimised.  The notation
$\NP(\qbf, L)$ is used to denote the nearest point in $L$ to $\qbf$.  We
assume that, when two or more lattice points are of equal distance to $\qbf$, $\NP(\qbf, L)$
selects one of the lattice points in a systematic manner.

The nearest lattice point problem is known to be NP-hard under certain conditions when the lattice itself, or rather a basis thereof, is considered as an additional input parameter \cite{micciancio_hardness_2001, AjtaiShortestVecProbNPHard1998, Jalden2005_sphere_decoding_complexity}. Nevertheless, algorithms exist that can compute the nearest lattice point in reasonable time if the dimension is small
\cite{Agrell2002, Viterbo_sphere_decoder_1999, Pohst_sphere_decoder_1981}. One
such algorithm introduced by Pohst \cite{Pohst_sphere_decoder_1981} in 1981
was popularised in the signal processing and communications fields by Viterbo
and Boutros \cite{Viterbo_sphere_decoder_1999} and has since been called the
\emph{sphere decoder}. Approximate algorithms for computing the nearest point
have also been studied.  One example is Babai's nearest plane algorithm
\cite{Babai1986}, which requires $O(n^4)$ arithmetic operations in the worst
case.  For specific lattices where the generator matrix is known a priori,
many fast nearest point algorithms are known \cite{Conway1982FastQuantDec,
  Conway1986SoftDecLeechGolay, Clarkson1999:Anstar, McKilliam2008,
  McKilliam2008b, McKilliam2009CoxeterLattices}.

We now show how the LSPUE can be represented as a nearest lattice point
problem in a lattice determined by $N$. Define the $N$-dimensional vectors
$\mathbf{n}=\left[1 \; 2 \; \cdots \; N\right]^{\prime}$, $\mathbf{1}=\left[1 \; 1 \; \cdots \; 1\right]^{\prime}$, $\mathbf{y}=\left[Y_1 \; Y_2 \; \cdots \; Y_N\right]^{\prime}$, $\mathbf{x}=\left[X_1 \; X_2 \; \cdots \; X_N\right]^{\prime}$ and $\mathbf{u}=\left[U_1 \; U_2 \; \cdots \; U_N\right]^{\prime}$. The sum of squares function
\eqref{eq:sumofsquaresfunctionwithU} can be written in vector form as
\begin{equation}\label{eq:vectorsumofsquares}
  \left\|\ybf - f\nbf - \theta\onebf - \ubf \right\|^2.
\end{equation}
Define the matrix $\Mbf = [\nbf \; \onebf].$ The least squares estimator is then
\begin{equation}\label{eq_least_squares_unwrapping_x}
\left[ \begin{array}[c]{c} \widehat{f}_N \\ \widehat{\theta}_N \end{array} \right] = \arg\min_{(f,\theta)\in[-\nicefrac{1}{2},\nicefrac{1}{2})^2}\min_{\ubf \in \ints^N }\left\|\ybf - \Mbf\left[ \begin{array}[c]{c} f \\ \theta \end{array} \right] - \ubf \right\|^2.
\end{equation}

Given $\ubf$, the least squares estimators of $f$ and $\theta$ are obtained by
the usual linear regression formulae and are given by
\begin{equation} \label{eq_regf} \left[ \begin{array}[c]{c} \widehat{f}_N \\
      \widehat{\theta}_N \end{array} \right] = \Mbf^+ (\ybf - \ubf),
\end{equation}
where $\Mbf^+ = (\Mbf'\Mbf)^{-1}\Mbf'$.  Substituting (\ref{eq_regf}) into
(\ref{eq_least_squares_unwrapping_x}) and rearranging, the least squares
estimator of $\ubf$, given $f$ and $\theta$, is
\begin{equation} \label{eq_minu}
\hat{\ubf} = \arg\min_{\ubf\in\ints^{N}} {\|\Bbf(\ybf - \ubf)\|},
\end{equation}
where $\Bbf = \Ibf - \Mbf\Mbf^+$ and $\Ibf$ is the $N \times N$ identity
matrix.

Let $\Lambda$ be the lattice with generator matrix $\Bbf$.  The least squares
phase unwrapping is then the $\hat{\ubf}$ for which $\Bbf\hat{\ubf}$ is the
nearest lattice point in $\Lambda$ to $\Bbf\ybf$.  The least squares estimate
(\ref{eq_least_squares_unwrapping_x}) can be computed by first computing
$\NP(\Bbf\ybf, \Lambda)$, producing both $\Bbf\hat{\ubf}$ and $\hat{\ubf}$.
$\widehat{f}_N$ and $\widehat{\theta}_N$ can then be computed using
\eqref{eq_regf}.

The most difficult part of the procedure is finding the nearest lattice point.
One possible solution is to use the sphere decoder.  Unfortunately the sphere
decoder has worst case exponential complexity and therefore is only
computationally feasible when $N$ is small.  However, the lattice $\Lambda$ is
not random and has a generator matrix with significant structure.  Using this
structure we can construct a polynomial-time algorithm to compute the nearest
point.  %In this paper we are interested in finding the estimates of $\hat{f}$ and $\hat{\theta}$ so we will define these algorithms to output these estimates rather than the nearest lattice point $\Bbf\hat{\ubf}$.  If the nearest lattice point is desired it can be calculated from $\hat{f}$ and $\hat{\theta}$ as $\Bbf\round{\ybf - \hat{f}\nbf - \hat{\theta}\onebf}$.

Note that the nearest point is given by $\Bbf\hat{\ubf}$ where
\begin{equation}\label{eq:nearest_point_minimisation}
\hat{\ubf} = \arg\min_{\ubf\in\ints^N}\min_{(f, \theta)\in[-\nicefrac{1}{2},\nicefrac{1}{2})^2} \|\ybf - f\nbf - \theta\onebf - \ubf\|^2 .
\end{equation}
Fixing both $f$ and $\ubf$ and minimising with respect to $\theta$ we obtain
\begin{equation} \label{eq:minwrttheta} 
\theta = \frac{\onebf'(\ybf - f\nbf - \ubf)}{N}.
\end{equation}
Substituting this into \eqref{eq:nearest_point_minimisation} we find that
\begin{equation} \label{eq:min_An*_line}
\hat{\ubf} = \arg\min_{\ubf\in\ints^N}\min_{f\in[-\nicefrac{1}{2},\nicefrac{1}{2})} \|\Qbf\ybf - f\Qbf\nbf - \Qbf\ubf\|^2 ,
\end{equation}
where $\Qbf$ is the projection matrix
\begin{equation} \label{eq:Q}
\Qbf =  \left(\Ibf - \frac{\onebf \onebf'}{\onebf'\onebf}\right).
\end{equation}
The matrix $\Qbf$ is the generator matrix for a well studied lattice called
$A_{N-1}^*$ \cite{SPLAG}.  Numerous nearest point algorithms exist for the
lattice $A_N^*$ \cite{Clarkson1999:Anstar, Conway1982FastQuantDec,
  McKilliam2008a, McKilliam2008b}.  The fastest known algorithm was described
by McKilliam \emph{et al.} and requires $O(N)$ arithmetic operations
\cite{McKilliam2008b}.

Let $\zbf = \Qbf\ybf$, $\gbf = \Qbf\nbf$ and $\zetabf = \Qbf\ubf$.  Given $f$,
\eqref{eq:min_An*_line} is minimised when
\[
\zetabf = \NP(\zbf - f\gbf, A_{N-1}^*).
\]
Conversely, given $\zetabf$, \eqref{eq:min_An*_line} is minimised by putting
\[
f = \frac{\gbf'(\zbf - \zetabf)}{\gbf'\gbf}.
\]
If we define the set
\begin{equation}\label{eq:setS}
  S = \left\{ \NP(\zbf - f\gbf, A_{N-1}^*) \; ; \; f\in[-\nicefrac{1}{2},\nicefrac{1}{2}) \right\},
\end{equation}
then the nearest point is given by $\Bbf\hat{\ubf}$ where $\hat{\ubf}$ satisfies
\begin{equation} \label{eq:findbinS}
\Qbf\hat{\ubf} = \arg\min_{\zetabf \in S} h(\zetabf)
\end{equation}
and $h: \reals^N \mapsto \reals$ is defined by
\[
h(\zetabf) = \left\|\zbf - \frac{\gbf'(\zbf - \zetabf)}{\gbf'\gbf}\gbf - \zetabf \right\|^2.
\]
The set $S$ contains the lattice points in $A_{N-1}^*$ that are nearest to
some point on the line segment defined by $\zbf - f\gbf$ where
$f\in[-\nicefrac{1}{2},\nicefrac{1}{2})$.

Note that $\#S$ is finite.  We require an algorithm to find the lattice points
in $S$.  The nearest point can then be found by testing each lattice point and
returning the minimiser according to (\ref{eq:findbinS}). Computing $S$
directly proves to be difficult, and, in fact, it is easier to compute a
superset of $S$.  To show this we require some properties of the lattice
$A_{N-1}^*$.  The generator matrix for $A_{N-1}^*$ is the matrix $\Qbf$
defined in \eqref{eq:Q}, which is the projection matrix into the hyperplane
orthogonal to $\onebf$.  Let $H$ denote this hyperplane.  Then $A_{N-1}^*$
consists of the lattice points in $\ints^N$ projected orthogonally onto $H$.
That is,
\[
A_{N-1}^* = \left\{ \Qbf\wbf \mid \wbf \in \ints^N \right\}.
\]
%It is easy to see that
%\begin{equation} \label{eq:VorAn*subsetvorZn}
%\vor(A_{N-1}^*) \cap H \subset \vor(\ints^N)
%\end{equation}

Another way of representing $A_{N-1}^*$ is as a union of $N$ translates of its dual lattice $A_{N-1} = \ints^{N} \cap H$ \cite{SPLAG}.  That is,
\begin{align}
  A_{N-1}^* &= \bigcup_{i=0}^{n-1} \left( [i] + A_{N-1} \right) \nonumber \\
  					&= \bigcup_{i=0}^{n-1} \left( [i] + \ints^N \right) \bigcap H \\
            &\subset \bigcup_{i=0}^{n-1} \left( [i] + \ints^N \right),\label{eq:An*asglues}
\end{align}
where the $[i]$ are known as \emph{glue vectors} and are defined in
this case as
\begin{equation} \label{eq:Anglues}
  [i] = \frac{1}{N} \big[ \underbrace{i, \dots, i}_{\text{$j$ times}},
        \underbrace{-j, \dots, -j}_{\text{$i$ times}}
  \big]
\end{equation}
for $i = 0,1, \dots, N-1$ with $i+j = N$ \cite[pp. 109]{SPLAG}.

There exists a fast algorithm to find the points in the set
\[
K(i) = \left\{ \NP(\zbf - f\gbf, [i] + \ints^N) \mid f \in [-\nicefrac{1}{2},\nicefrac{1}{2}) \right\}
\]
due to Ryan \emph{et al.} \cite{Ryan2007, McKilliam2007}.  We use this algorithm for each $i = 0,1,\dots,N-1$. Let
\begin{equation} \label{eq:S+}
S^+ = \bigcup_{i=0}^{n-1} K(i).
\end{equation}

The proof of the following lemma is not difficult and is omitted.

\begin{lemma} \label{lem:minSS+equal}
\[
\Qbf\hat{\ubf} = \arg\min_{\zetabf \in S} h(\zetabf) = \arg\min_{\zetabf \in S^+} h(\zetabf).
\]
\end{lemma}

It is thus sufficient to use the set $S^+$ rather than $S$ to find the nearest lattice point.  Since $S^+$ is the union of the $K(i)$,
\[
\Qbf\hat{\ubf} = \arg\min_{\zetabf \in K(i)}\min_{i=0,1,\dots,N-1} h(\zetabf).
\]
For each $i$ the minimisation can be computed using the algorithm in \cite{Ryan2007, McKilliam2007}.  The number of operations required is $O(\#S^+\log{N})$.

\begin{lemma} \label{lem:sizeK=O(N^2)}
For $i=0,1,\dots,N$
\[
\#K(i) = O(N^2).
\]
\end{lemma}
\begin{IEEEproof}
The `first' lattice point in $K(i)$ corresponds to $f = -\nicefrac{1}{2}$, and is given by
\[
\vbf = \round{\zbf + \frac{\gbf}{2} - [i]} + [i],
\]
while the `last' lattice point in $K(i)$ corresponds to $f = \nicefrac{1}{2}$, and is given by
\[
\wbf = \round{\zbf - \frac{\gbf}{2} - [i]} + [i].
\]
Consecutive elements in $K(i)$ satisfy the equation
\[
\vbf_{\text{next}} = \vbf_{\text{previous}} + \sign{g_j}\ebf_j 
\]
for some $j$, where $\ebf_j$ is a vector conistsing of zeros apart from a one in the $j$th position and $\sign{g_j}$ is $1$ if $g_j > 0$, $-1$ if $g_j < 0$ and $0$ if $g_j = 0$.  Thus
\[
\#K(i) = \sum_{n=1}^N |v_n - w_n| \leq \sum_{n=1}^N |w_n| + |v_n|.
\]
Since $z_n$ and $[i]_n \in [-\nicefrac{1}{2}, \nicefrac{1}{2})$, and $|g_n| \leq N$, $|v_n|,|w_n| \leq N+2$ for all $n=1,2,\dots,N$.  Hence
\[
\#K(i) \leq \sum_{n=1}^N{ \left(2N + 4\right) } = O(N^2).
\] 
\end{IEEEproof}

As a consequence of the lemma, $\#S^+ = O(N^3)$ and the algorithm requires $O(N^3\log{N})$ arithmetic operations. 


% \begin{algorithm} \label{alg:ON^3logN}
% \SetAlCapFnt{\small}
% \SetAlTitleFnt{}
% \dontprintsemicolon
% $\zbf = \Qbf\ybf$ \;
% $\gbf = \Qbf\nbf$ \;
% $D = \infty$ \;
% \For{$i = 0$ \emph{\textbf{to}} $n-1$}{ 
% 	$\vbf = \round{\zbf - [i]} + [i]$\;
% 	\For{$t = 1$ \emph{\textbf{to}} $n$ \nllabel{alg:alg_for_glue}}{
% 		\If{$g_t \neq 0$}{
% 			$f^* = \frac{z_t - v_t + 0.5\sign{g_t}}{g_t}$ \;
% 			$\operatorname{AddToSortedList}(f^*, t)$ \;
% 		}
% 	}
% 	$\alpha = \gbf'(\vbf - \zbf)$ \;
% 	$\beta = \|\zbf - \vbf\|^2$ \;
% 	$\gamma = \|\gbf\|^2$ \;
% 	$f = \frac{\gbf'(\zbf - \vbf)}{\gbf'\gbf}$ \;
% \While{$f < 1$ \nllabel{alg:whilef<1}}{
% 	$h =  \beta + 2f\alpha + f^2\gamma$ \;\
% 	\If{$h < D$ \nllabel{snpe:alg:alg_ifL}}{
% 		$D = h$ \;
% 		$\hat{f} = f$ \;
% 	}
% 	$t = \operatorname{RemoveFirstFromSortedList}$ \;
% 	$\alpha = \alpha - |g_{t}|$ \;
% 	$\beta = \beta + 2\sign{g_{t}} \left( z_{t} - v_{t} \right) + 1$\;
% 	$f = f + \frac{|g_{t}|}{\|\gbf\|^2}$ \;
% 	$v_{t} = v_{t} - \sign{g_t}$\; 
% 	$f^* = \frac{z_t - v_t + 0.5\sign{g_t}}{g_t}$ \;
% 	$\operatorname{AddToSortedList}(f^*, t)$ \;
% 	}
% }
% $(\zetabf, \hat{\ubf}) = \NP(\zbf - \hat{f}\gbf, A_{N-1}^*)$ \;
% \Return{$(\Bbf\hat{\ubf}, \hat{\ubf})$ \nllabel{snpe:alg:alg_returnT}}
% \caption{$O(N^3\log{N})$ algorithm to find the nearest lattice point}
% \end{algorithm}

%%% Local Variables: 
%%% mode: latex
%%% TeX-master: "lsuwrapping.tex"
%%% End:
