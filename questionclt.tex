\documentclass{article}

\usepackage{ifpdf}
\ifpdf
  \usepackage[pdftex]{graphicx}
  \usepackage{thumbpdf}
  \usepackage[naturalnames]{hyperref}
  \DeclareGraphicsRule{*}{mps}{*}{}
\else
	\usepackage{graphicx}
\fi
\usepackage{bm, booktabs}
\usepackage{amsmath}%
\usepackage{amsfonts}%
\usepackage{amssymb}%
\usepackage{amsthm}
\usepackage{mathrsfs}
%\usepackage{graphicx}
%\usepackage{setspace}
\usepackage{cite}
%\usepackage{times}
\usepackage[normalem]{ulem}
%\usepackage{algorithmic}
%\usepackage[figure, vlined, linesnumbered]{algorithm2e}
\usepackage[vlined, linesnumbered]{algorithm2e}
%\usepackage{algorithm}
\usepackage{units}

\newcommand{\round}[1]{\lfloor #1 \rceil}
\newcommand{\floor}[1]{\lfloor #1 \rfloor}
\newcommand{\ceil}[1]{\lceil #1 \rceil}
\newcommand{\reals}{{\mathbb R}}
\newcommand{\ints}{{\mathbb Z}}
\newcommand{\integers}{{\mathbb Z}}
\newcommand{\uy}{\underline{\bm{Y}}}
\newcommand{\uey}{\underline{Y}}
\newcommand{\sign}[1]{\mathtt{sign}(#1)}
\newcommand{\Qbf}{{\mathbf Q}}
\newcommand{\Bbf}{{\mathbf B}}
\newcommand{\Ibf}{{\mathbf I}}
\newcommand{\ybf}{{\mathbf y}}
\newcommand{\xbf}{{\mathbf x}}
\newcommand{\zbf}{{\mathbf z}}
\newcommand{\ebf}{{\mathbf e}}
\newcommand{\vbf}{{\mathbf v}}
\newcommand{\wbf}{{\mathbf w}}
\newcommand{\kbf}{{\mathbf k}}
\newcommand{\sbf}{{\mathbf s}}
\newcommand{\Pibf}{{\mathbf \Pi}}
\newcommand{\onebf}{{\mathbf 1}}
\newcommand{\fbf}{{\mathbf f}}
\newcommand{\ubf}{{\mathbf u}}

\newtheorem{theorem}{Theorem}
%\theoremstyle{plain}
\newtheorem{acknowledgement}{Acknowledgement}
%\newtheorem{algorithm}{Algorithm}
\newtheorem{axiom}{Axiom}
\newtheorem{case}{Case}
\newtheorem{claim}{Claim}
\newtheorem{conclusion}{Conclusion}
\newtheorem{condition}{Condition}
\newtheorem{conjecture}{Conjecture}
\newtheorem{corollary}{Corollary}
\newtheorem{criterion}{Criterion}
\newtheorem{definition}{Definition}
\newtheorem{example}{Example}
\newtheorem{exercise}{Exercise}
\newtheorem{lemma}{Lemma}
\newtheorem{notation}{Notation}
\newtheorem{problem}{Problem}
\newtheorem{proposition}{Proposition}
\newtheorem{remark}{Remark}
\newtheorem{solution}{Solution}
\newtheorem{summary}{Summary}

\newcommand{\vol}{\operatorname{vol}}
\newcommand{\vor}{\operatorname{Vor}}

\begin{document}

In the proof of the CLT we find that
$
N^{1/2}\left[\begin{array} 
[c]{c}%
\psi_{N}\\
\phi_{N}%
\end{array}
\right]
$
is normally distributed with mean 0 and covariance matrix
\[
\frac{12\sigma^{2}}{\left(  1-h  \right)  ^{2}}\left[
\begin{array}
[c]{cc}%
1 & -1/2\\
-1/2 & 1/3
\end{array}
\right] = \mathbf{M}
\]
and therefore
$
\left[\begin{array} 
[c]{c}%
\psi_{N}\\
\phi_{N}%
\end{array}
\right]
$
is normally distributed with mean 0 and covariance matrix $N^{-1}\mathbf{M}$. We then show that
\[
\left[
\begin{array}
[c]{c}%
\psi\\
\phi
\end{array}
\right]  =\left[
\begin{array} 
[c]{c}%
\psi_{N}\\
\phi_{N}%
\end{array}
\right]  \left(  1+o_{P}\left(  1\right)  \right)  .
\]
and because convergence in probability implies convergence in distribution it follows that $\left[
\begin{array}
[c]{c}%
\psi\\
\phi
\end{array}
\right]$
is normally distributed with covariance matrix $N^{-1}\mathbf{M}$.  However, we do not show that
$N^{1/2}\left[\begin{array}
[c]{c}%
\psi\\
\phi
\end{array}
\right]$
converges to $N^{1/2}\left[
\begin{array} 
[c]{c}%
\psi_{N}\\
\phi_{N}%
\end{array}
\right]$
in probability and therefore cannot say that $N^{1/2}\left[\begin{array}
[c]{c}%
\psi\\
\phi
\end{array}
\right]$ is normally distributed with mean 0 and covariance $\mathbf{M}$.

\end{document}
