\section{Introduction}

Estimation of the frequency of a single noisy sinusoid is a long studied
problem with applications including radar, sonar, telecommunications,
astronomy and medicine \cite{Quinn2001, Rife1974}.  In this paper, a single
frequency signal is modelled as a complex sinusoid of the form
\begin{equation} \label{eq_sinusoidal}
A \exp{\big(2\pi j \left(f_0 n + \theta_0 \right)\big)},
\end{equation}
where $f_0$ is the frequency, $\theta_0$ is the phase, $n \in \{1,2\dots,N\}$, $A$
is the signal amplitude and $j = \sqrt{-1}$.  The aim is to estimate the
parameters $f_0$ and $\theta_0$ from the signal
\begin{equation} \label{eq_sigmodel_noise}
v_n = A \exp{\big(2\pi j \left(f_0 n + \theta_0 \right)\big)} + s_n,
\end{equation} 
where the sequence $s_1, s_2, \dots$ is a complex noise process. We shall assume, in
this paper, that the random variables $s_n$ are independent and identically
distributed, and that the distribution of $e^{j \alpha} s_n$ does not depend
on $\alpha$. This will occur exactly when the distribution of $s_n$
depends only on $\vert s_n \vert$.  To ensure identifiability we assume that
$f_0$ and $\theta_0$ are in $[-\nicefrac{1}{2}, \nicefrac{1}{2})$.

The maximum likelihood estimator of frequency under Gaussian white assumptions
is known to be very closely approximated by the frequency that maximises the
\emph{periodogram} (squared magnitude of the Fourier transform) of the $v_n$
\cite{Quinn2001}.
%That is 
%\begin{equation} \label{eq:periodogram}
%\hat{f} = \arg\max_{f\in[0,1)}{ P_v(f) }
%\end{equation}
%where
%\[
%P_v(f) = \frac{1}{N} \left| \sum_{n=1}^{N} v_n e^{2\pi j f n}  \right|^2
%\]
We refer to this estimator as the \emph{periodogram estimator}; its
asymptotic properties have been known for some
time \cite{Quinn2001, Hannan1973, Walker1971}.

Rife and Boorstyn have suggested a practical method for computing the
periodogram estimator, by using the fast Fourier transform to obtain the value
of the periodogram at the Fourier frequencies $f =
-\nicefrac{1}{2},-\nicefrac{1}{2}+\nicefrac{1}{N},-\nicefrac{1}{2}+\nicefrac{2}{N},
\dots, \nicefrac{1}{2}-\nicefrac{1}{N}$ \cite{Rife1974}.  The Fourier
frequency that maximises the periodogram is found and this estimate is then
further refined by a numerical approach such as Newton's method.  A problem
with this approach is that the numerical procedure can fail to locate the
correct maximizer \cite{Rice1988}.  To avoid the problem, Rife and Boorstyn
suggested zero padding the signal to the length $4N$ before performing the
fast Fourier transform. This has recently been shown to work by Quinn \emph{et
  al.}, who also show that applying Newton's method to the derivative of
certain monotonic functions of the periodogram, rather than to the periodogram
itself, ensures that Newton's method will succeed even without any zero
padding \cite{Quinn2008maximizing_the_periodogram}.

Regardless of these implementation difficulties, the periodogram estimator is
widely seen as the best method for single frequency estimation.  It provides
the most accurate results and using the fast Fourier transform can be computed
in only $O(N\log{N})$ arithmetic operations.  Nevertheless, many other methods
for single frequency estimation exist.

One alternative method is \emph{phase unwrapping}~\cite{Fowler2002_freq_est_by_phases}.  Phase unwrapping
estimators appear to have been first suggested by Tretter~\cite{Tretter1985},
who utilized the fact that the phase of a complex sinusoid is a linear
function that is `wrapped' modulo $2\pi$.  %This is
%\[
%\arg\left\{ \exp{\left(2\pi j \left(f_0 n + \theta_0 \right)\right)} \right\} = 2\pi(f_0 n + \theta_0) 
%\]
%modulo $2\pi$.  
If the phase could be `unwrapped' then $f_0$ and $\theta_0$ could be estimated
by linear regression.  Many phase unwrapping estimators have since been
suggested \cite{Kay1989, Clarkson1994AnalysisKaysVariance, Clarkson1999,
  Lovell1991, Lovell1992_stat_perform_inst_freq_est}. A common approach is to compute the first differences of the
arguments of the $v_n$.  The resulting signal then resembles a moving average
process, whose parameters can be estimated by standard linear techniques.
This approach was first suggested by Kay \cite{Kay1989,
  Clarkson1994AnalysisKaysVariance}.  A significant advantage of this approach
is that the moving average process has enough structure for the estimates to
be computed with only $O(N)$ arithmetic operations.  The estimator also
appears to perform well when the signal-to-noise ratio is large.  The major
drawback of Kay's estimator is that it thresholds at a relatively high SNR and
also performs poorly at moderate SNRs \cite{Clarkson1994}. In fact, the
estimator has been shown to be inconsistent, i.e., converges almost surely as
$N \rightarrow \infty$ to the wrong frequency \cite{Quinn2001}.


Another approach is to unwrap the phase in the least squares sense.  We call
the resulting estimator the least squares phase unwrapping estimator (LSPUE).
Clarkson has shown how the LSPUE is related to a problem in algorithmic number
theory known as the nearest lattice point problem \cite{Clarkson1999,
  Agrell2002}.  He has also shown that the LSPUE is closely related to the
periodogram estimator.  Quinn has derived the asymptotic properties of a LSPUE
in the simpler case of phase estimation \emph{i.e.} when $f_0 = 0$ and the aim
is to estimate $\theta_0$ \cite{Quinn2007}.

In this paper we derive the asymptotic properties of the LSPUE for single
frequency estimation (Section \ref{sec:asymptotic_properties}).  We show that
the estimator is strongly consistent and derive its central limit
theorem. Following Clarkson \cite{Clarkson1999} we discuss methods to compute
the LSPUE using a nearest lattice point approach (Section
\ref{sec:LSPUandNP}).  While Clarkson suggested solving the nearest lattice
point problem using the approximate algorithm of Babai \cite{Babai1986}, we
describe a polynomial-time algorithm that computes the estimator exactly.  In
Section \ref{sec:simulations} we provide simulations that display the
statistical performance of the LSPUE alongside the periodogram estimator and
some other single frequency estimators.  The LSPUE proves to be only
marginally less accurate than the periodogram estimator and significantly more
accurate than other estimators based on phase unwrapping. We also use the
simulations to confirm the asymptotic results derived in Section
\ref{sec:asymptotic_properties}.


%\section{Signal model}
%
%A single frequency signal is modelled as a complex sinusoid of the form
%\begin{equation} \label{eq_sinusoidal}
%A \exp{\left(2\pi j \left(f_0 n + \theta_0 \right) \right)}
%\end{equation}
%where $n = 1,2\dots,N$, $f_0$ and $\theta_0$ are in [-\nicefrac{1}{2}, \nicefrac{1}{2}), $A$ is the signal amplitude and $j = \sqrt{-1}$.  The aim is to estimate the parameters $f_0$ and $\theta_0$ from the signal
%\begin{equation} \label{eq_sigmodel_noise}
%v_n = A \exp{\left(2\pi j \left(f_0 n + \theta_0 \right) \right)} + s_n
%\end{equation} 
%where $s_n$ is a complex noise process.


\section{The least squares phase unwrapping estimator}
The argument of the $v_n$, denoted $\angle{v_n}$, is given by
\[
\angle{v_n} = 2\pi\left( f_0 n + \theta_0 + X_n \right) \pmod{2\pi},
\]
where $X_n$ is noise in the range $[-\nicefrac{1}{2}, \nicefrac{1}{2})$ and 
has a distribution depending only on $A$ and the probability density function
(pdf) of the $s_n$.  When $s_n$ is complex Gaussian noise the distribution of
the $X_n$ is known as the projected normal distribution and has been studied
by Mardia and Jupp \cite[p. 46]{Mardia_directional_statistics}.  The
distribution has also been discussed by Quinn \cite{Quinn2007} and Tretter
\cite{Tretter1985}.  Other circular noise distributions may be used, for
example, the wrapped normal, or von Mises distributions \cite{Fisher1993,
  Mardia_directional_statistics}.  In this paper we assume that the $X_n$ are
continuous, independent and identically symmetrically distributed, with
cumulative distribution function (cdf) $F_X\left( x\right) $ and pdf $f_X
\left( x\right)$.  We denote by $\sigma^{2}$ the common variance of the $X_n$.
We also assume that $f_X \left( x\right) $ is unimodal, with mode at
$0$. %Indeed, without such an assumption, the parameter $\theta_0$ is not properly identified.
These assumptions are satisfied by a wide range of circular distributions
including the projected normal, wrapped normal and von Mises distributions
\cite{Fisher1993, Mardia_directional_statistics}

Let $Y_n = \angle{v_n}/ \left(2\pi\right)$.  Then
\begin{equation} \label{eq_polyphasemodel}
Y_n = \fracpart{ f_0 n + \theta_0 + X_n },
\end{equation}
where $\fracpart{x} = x - \round{x}$ is the fractional part of $x$ and $\round{x}$
denotes the nearest integer to $x$\footnote{The direction of rounding for
  half-integers is not important, as long as it is consistent. The authors
  have chosen to round up half-integers here.}.

We may write \eqref{eq_polyphasemodel} as
\begin{equation} \label{eq:eq_polyphasemodel}
Y_n = f_0 n + \theta_0 + X_n + U_n,
\end{equation}
where $U_n = -\round{f_0 n + \theta_0 + X_n}$.  By considering the $U_n$ as
nuisance parameters, where $U_n \in \ints$, we may derive the sum of squares
function
\begin{equation} \label{eq:sumofsquaresfunctionwithU}
\sum_{n=1}^{N}\left(  Y_{n} - \theta - fn - U_{n}\right)^{2}.
\end{equation}
For fixed $f$ and $\theta$, \eqref{eq:sumofsquaresfunctionwithU} is minimized
when $U_{n} = -\round{Y_{n} - \theta - fn}$.  Substituting this into
\eqref{eq:sumofsquaresfunctionwithU} we obtain
\begin{equation} \label{eq:sumofsquaresfunction}
SS(f, \theta) = \sum_{n=1}^{N}\fracpart{  Y_{n} - \theta - fn }^{2}.
\end{equation}
The LSPUE returns the $f$ and $\theta$ that minimise $SS(f, \theta)$.


%%% Local Variables: 
%%% mode: latex
%%% TeX-master: "lsuwrapping.tex"
%%% End: 
